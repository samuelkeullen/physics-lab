\documentclass[12pt]{article}
\usepackage{amsmath, amssymb, physics, geometry, hyperref, xcolor, siunitx}
\geometry{a4paper, margin=2.2cm}
\sisetup{output-exponent-marker=\ensuremath{\mathrm{e}}}

\title{Métodos de Aproximação Quântica: História, Formalismo e Exemplos Numéricos Detalhados}
\author{Samuel Keullen Sales}
\date{\today}

\begin{document}
\maketitle

\section{Introdução}
Este documento apresenta três métodos centrais de aproximação em mecânica quântica: teoria de perturbação, método variacional e aproximação WKB. Para cada método, são fornecidas notas históricas, definições formais e fórmulas (com legendas), aplicações práticas e exemplos resolvidos com cálculos numéricos detalhados passo a passo.

\section{Convenções e constantes}
Salvo indicação contrária, são usadas unidades SI. Constantes principais:
\[
\hbar = 1.054571817\times 10^{-34}\ \text{J·s},\qquad
m_e = 9.10938356\times 10^{-31}\ \text{kg},
\]
\[
e = 1.602176634\times 10^{-19}\ \text{C},\qquad
\varepsilon_0 = 8.8541878128\times 10^{-12}\ \text{F/m},
\]
definindo
\[
k\equiv\frac{1}{4\pi\varepsilon_0}\approx 8.9875517923\times 10^{9}\ \text{N·m}^2\text{/C}^2.
\]

\section{Teoria de Perturbação}

\subsection{História}
Desenvolvida e formalizada entre 1926--1927 por Schrödinger e Dirac. Essencial para correções pequenas em sistemas cujos Hamiltonianos são exatamente solúveis.

\subsection{Formalismo}
Dado $\hat H=\hat H_0+\lambda \hat H'$, onde $\hat H_0$ é solúvel, expande-se energia e estado:
\[
E_n = E_n^{(0)} + \lambda E_n^{(1)} + \lambda^2 E_n^{(2)} + \cdots,
\]
\[
\ket{\psi_n} = \ket{\psi_n^{(0)}} + \lambda \ket{\psi_n^{(1)}} + \lambda^2 \ket{\psi_n^{(2)}} + \cdots.
\]
Correções de primeira e segunda ordem:
\[
E_n^{(1)} = \bra{\psi_n^{(0)}}\hat H'\ket{\psi_n^{(0)}},\qquad
E_n^{(2)} = \sum_{m\ne n}\frac{|\bra{\psi_m^{(0)}}\hat H'\ket{\psi_n^{(0)}}|^2}{E_n^{(0)}-E_m^{(0)}}.
\]

\subsection{Legenda}
\begin{itemize}
  \item $\hat H_0$ : Hamiltoniano não perturbado
  \item $\hat H'$ : operador de perturbação
  \item $\lambda$ : parâmetro formal (ajustado a 1 no fim)
  \item $E_n^{(k)}$ : correção de $k$-ésima ordem para o nível $n$
  \item $\ket{\psi_n^{(0)}}$ : autoestado não perturbado
\end{itemize}

\subsection{Exemplo 1: Oscilador harmônico com perturbação cúbica}

\[
\hat H = \hat H_0 + \lambda \hat x^3,\qquad
\hat H_0=\frac{\hat p^2}{2m}+\frac{1}{2}m\omega^2\hat x^2.
\]
Calcular $E_0^{(1)}$ e $E_0^{(2)}$ e estimar numericamente para $\lambda=0.01$.

\subsubsection{Base não perturbada}
\[
E_n^{(0)}=\hbar\omega\left(n+\tfrac12\right),\qquad
\hat x = \sqrt{\frac{\hbar}{2m\omega}}\,(a+a^\dagger).
\]
Logo:
\[
\hat x^3 = \left(\sqrt{\frac{\hbar}{2m\omega}}\right)^3 (a+a^\dagger)^3.
\]

\subsubsection{Expansão}
\[
(a+a^\dagger)^3 = a^3 + 3 a^2 a^\dagger + 3 a (a^\dagger)^2 + (a^\dagger)^3.
\]
Aplicando em $\ket{0}$:
\[
(a^\dagger)^2\ket{0}=\sqrt{2}\ket{2},\quad (a^\dagger)^3\ket{0}=\sqrt{6}\ket{3},\quad a(a^\dagger)^2\ket{0}=2\ket{1}.
\]
Portanto:
\[
\hat x^3\ket{0}=\left(\sqrt{\frac{\hbar}{2m\omega}}\right)^3(6\ket{1}+\sqrt{6}\ket{3}).
\]
Primeira ordem:
\[
E_0^{(1)}=0.
\]
Segunda ordem:
\[
E_0^{(2)} = \lambda^2\left[
\frac{36\left(\frac{\hbar}{2m\omega}\right)^3}{-\hbar\omega}+
\frac{6\left(\frac{\hbar}{2m\omega}\right)^3}{-3\hbar\omega}
\right]
= -38\,\lambda^2\,\frac{\hbar^2}{8\,m^3\omega^4}.
\]

\subsubsection{Exemplo numérico}
Usando $\hbar=1$, $m=1$, $\omega=1$, $\lambda=0.01$:
\[
E_0^{(2)}=-38(0.01)^2=-0.0038,\qquad E_0\approx0.5-0.0038=0.4962.
\]

\subsubsection{Exemplo físico}
Usando valores reais:
\[
m=m_e,\ \omega=1\times10^{15}\ \text{rad/s}.
\]
\[
\frac{\hbar}{2m\omega}=\frac{1.054571817\times10^{-34}}{1.821876712\times10^{-15}}\approx5.788\times10^{-20}.
\]
\[
\left(\frac{\hbar}{2m\omega}\right)^3\approx(5.788\times10^{-20})^3=1.93898\times10^{-58}.
\]
\[
\frac{\left(\frac{\hbar}{2m\omega}\right)^3}{\hbar\omega}=\frac{1.93898\times10^{-58}}{1.054571817\times10^{-19}}\approx1.8387\times10^{-39}.
\]
Logo:
\[
E_0^{(2)}=-38\lambda^2\times1.8387\times10^{-39}=-6.988\times10^{-38}\lambda^2.
\]
Para $\lambda=0.01$:
\[
E_0^{(2)}=-6.988\times10^{-42}\ \text{J}\approx-4.36\times10^{-23}\ \text{eV}.
\]

\section{Método Variacional}

\subsection{História}
Baseado no princípio de Rayleigh, aplicado à mecânica quântica por Heisenberg e Schrödinger (1926).

\subsection{Formalismo}
Para qualquer função de teste normalizada $\phi(\mathbf{r};\{\alpha_i\})$:
\[
E_0 \le E[\phi] = \frac{\bra{\phi}\hat H\ket{\phi}}{\braket{\phi|\phi}}.
\]
Minimiza-se $E[\phi]$ em relação aos parâmetros variacionais $\{\alpha_i\}$.

\subsection{Exemplo: Átomo de hidrogênio com função exponencial}
\[
\hat H=-\frac{\hbar^2}{2m_e}\nabla^2-\frac{k e^2}{r},\quad
\phi(r)=\sqrt{\frac{\alpha^3}{\pi}}e^{-\alpha r}.
\]
\[
E(\alpha)=\frac{\hbar^2\alpha^2}{2m_e}-k e^2\alpha.
\]
Condição de mínimo:
\[
\frac{dE}{d\alpha}=0\Rightarrow \alpha_{\text{opt}}=\frac{m_e k e^2}{\hbar^2}.
\]
\[
E_0=-\frac{m_e(k e^2)^2}{2\hbar^2}.
\]

\subsubsection{Cálculo numérico}
\[
e^2=(1.602176634\times10^{-19})^2=2.56697\times10^{-38},
\]
\[
k e^2=8.98755\times10^{9}\times2.56697\times10^{-38}=2.30719\times10^{-28},
\]
\[
\hbar^2=(1.05457\times10^{-34})^2=1.11212\times10^{-68},
\]
\[
m_e k e^2=9.10938\times10^{-31}\times2.30719\times10^{-28}=2.1012\times10^{-58},
\]
\[
\alpha_{\text{opt}}=\frac{2.1012\times10^{-58}}{1.11212\times10^{-68}}=1.8897\times10^{10}\ \text{m}^{-1}.
\]
\[
E_0=-\frac{4.8482\times10^{-86}}{2.2242\times10^{-68}}=-2.1799\times10^{-18}\ \text{J}=-13.6057\ \text{eV}.
\]

\section{Aproximação WKB}

\subsection{História}
Desenvolvida independentemente por Wentzel, Kramers e Brillouin (1926). Importante em potenciais lentos e tunelamento quântico.

\subsection{Formalismo}
\[
-\frac{\hbar^2}{2m}\psi''(x)+V(x)\psi(x)=E\psi(x),
\]
\[
\psi(x)\approx\frac{C}{\sqrt{p(x)}}\exp\left(\pm\frac{i}{\hbar}\int^x p(x')\,dx'\right),
\quad p(x)=\sqrt{2m(E-V(x))}.
\]
Condição de quantização:
\[
\int_{x_1}^{x_2}p(x)\,dx=\left(n+\tfrac12\right)\pi\hbar.
\]

\subsection{Exemplo: Poço infinito de largura $L=1\,\text{nm}$}
\[
E_n=\frac{(n+\tfrac12)^2\pi^2\hbar^2}{2mL^2}.
\]
\[
m=m_e=9.10938\times10^{-31}\ \text{kg},\quad
\hbar=1.05457\times10^{-34},\quad
L=1.00\times10^{-9}\ \text{m}.
\]

\subsubsection{Nível fundamental ($n=0$)}
\[
(n+\tfrac12)\pi\hbar=0.5\times3.1416\times1.0546\times10^{-34}=1.6574\times10^{-34}.
\]
\[
\sqrt{2mE}=\frac{1.6574\times10^{-34}}{1\times10^{-9}}=1.6574\times10^{-25}.
\]
\[
E_0=\frac{(1.6574\times10^{-25})^2}{2\times9.10938\times10^{-31}}=1.507\times10^{-20}\ \text{J}=0.0940\ \text{eV}.
\]

\subsubsection{Primeiro excitado ($n=1$)}
\[
1.5\pi\hbar=4.972\times10^{-34},\quad
E_1=\frac{(4.972\times10^{-25})^2}{1.8219\times10^{-30}}=1.357\times10^{-19}\ \text{J}=0.846\ \text{eV}.
\]

\section{Tabela comparativa}
\begin{center}
\begin{tabular}{|l|p{6.5cm}|p{5.5cm}|}
\hline
Método & Aplicação típica & Vantagens / Limitações \\
\hline
Teoria de perturbação & Correções pequenas a sistemas conhecidos (campos externos, anarmonicidades) & Expansão sistemática; falha se a perturbação não for pequena \\
\hline
Método variacional & Estimativas do estado fundamental (moléculas, muitos corpos) & Dá limite superior da energia; depende da função de teste \\
\hline
WKB & Quantização semiclassical e tunelamento & Intuitivo; falha em potenciais abruptos ou baixos $n$ \\
\hline
\end{tabular}
\end{center}

\section{Conclusões}
Os exemplos apresentados demonstram os métodos perturbativo, variacional e WKB com derivação algébrica e cálculo numérico detalhado. Recomenda-se repetir os cálculos variando parâmetros ($L$, $\omega$, $\lambda$) para observar o comportamento escalar e fixar os conceitos.

\end{document}
