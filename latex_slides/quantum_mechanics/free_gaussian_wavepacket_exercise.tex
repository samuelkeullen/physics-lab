\documentclass[a4paper,12pt]{article}
\usepackage[utf8]{inputenc}
\usepackage{amsmath, amssymb}
\usepackage{geometry}
\geometry{margin=2cm}
\title{Exemplo Resolvido: Pacote Gaussiano Livre}
\author{Samuel Keullen Sales}
\date{\today}

\begin{document}
\maketitle

\section*{Exercício Resolvido: Evolução de um Pacote Gaussiano Livre}

Um elétron de massa $m_e = 9.109\times10^{-31}\,\text{kg}$ está inicialmente localizado em $x=0$ com uma função de onda gaussiana de largura $\sigma_0$ e número de onda central $k_0$.  
A função de onda inicial é:

\[
\psi(x,0) = \left(\frac{1}{\pi \sigma_0^2}\right)^{1/4} e^{-x^2/(2\sigma_0^2)} e^{i k_0 x}
\]

\textbf{Dados numéricos:}
\[
\sigma_0 = 2.0\times10^{-10}\,\text{m}, \quad
k_0 = 5.0\times10^9\,\text{m}^{-1}, \quad
t = 2.0\times10^{-15}\,\text{s}, 
\]
\[
\hbar = 1.054\times10^{-34}\,\text{J·s}, \quad
m_e = 9.109\times10^{-31}\,\text{kg}.
\]

\section*{1. Função de onda inicial}

\textbf{(a) Cálculo do fator de normalização:}

\[
A = \left(\frac{1}{\pi\sigma_0^2}\right)^{1/4}
\]
\[
\sigma_0^2 = (2.0\times10^{-10})^2 = 4.0\times10^{-20}
\]
\[
\pi\sigma_0^2 = 3.1416 \times 4.0\times10^{-20} = 1.25664\times10^{-19}
\]
\[
\frac{1}{\pi\sigma_0^2} = 7.95775\times10^{18}
\]
\[
A = (7.95775\times10^{18})^{1/4} = 5.320\times10^4 \quad [\text{m}^{-1/2}]
\]

\textbf{(b) Substituindo os valores:}

\[
\psi(x,0) = 5.320\times10^4 \, e^{-x^2/(8.0\times10^{-20})} e^{i(5.0\times10^9)x}
\]

\section*{2. Momento e velocidade}

\textbf{(a) Momento médio:}
\[
p_0 = \hbar k_0
\]
\[
p_0 = (1.054\times10^{-34})(5.0\times10^9) = 5.27\times10^{-25}\, \text{kg·m/s}
\]

\textbf{(b) Velocidade média:}
\[
v = \frac{p_0}{m_e}
\]
\[
v = \frac{5.27\times10^{-25}}{9.109\times10^{-31}} = 5.7847\times10^5\, \text{m/s}
\]

\section*{3. Evolução temporal}

\textbf{(a) Cálculo do fator complexo:}
\[
\tau = \frac{\hbar t}{2 m_e \sigma_0^2}
\]
\[
\hbar t = (1.054\times10^{-34})(2.0\times10^{-15}) = 2.108\times10^{-49}
\]
\[
2 m_e = 2 \times 9.109\times10^{-31} = 1.8218\times10^{-30}
\]
\[
2 m_e \sigma_0^2 = (1.8218\times10^{-30})(4.0\times10^{-20}) = 7.2872\times10^{-50}
\]
\[
\tau = \frac{2.108\times10^{-49}}{7.2872\times10^{-50}} = 2.892743
\]
\[
i\frac{\hbar t}{2m_e\sigma_0^2} = i\,2.892743
\]

\textbf{(b) Denominador complexo:}
\[
1 + i\frac{\hbar t}{2m_e\sigma_0^2} = 1 + i\,2.892743
\]

\textbf{(c) Função de onda no tempo $t$:}
\[
\psi(x,t) = A \left(1 + i\frac{\hbar t}{2m_e\sigma_0^2}\right)^{-1/2}
e^{-\frac{(x - vt)^2}{4\sigma_0^2(1 + i\frac{\hbar t}{2m_e\sigma_0^2})}}
e^{i\frac{p_0x}{\hbar}} e^{-i\frac{p_0^2t}{2m_e\hbar}}
\]

Substituindo valores:
\[
\psi(x,t) = 5.320\times10^4 (1 + i\,2.892743)^{-1/2}
e^{-\frac{(x - 5.7847\times10^5 t)^2}{4(4.0\times10^{-20})(1 + i\,2.892743)}}
e^{i(5.0\times10^9)x} e^{-i\,2.8927}
\]

\section*{4. Densidade de probabilidade}

\textbf{(a) Largura no tempo:}
\[
\sigma_t = \sigma_0 \sqrt{1 + \left(\frac{\hbar t}{2m_e\sigma_0^2}\right)^2}
\]
\[
\tau^2 = (2.892743)^2 = 8.36894
\]
\[
1 + \tau^2 = 9.36894
\]
\[
\sqrt{1 + \tau^2} = 3.0616
\]
\[
\sigma_t = (2.0\times10^{-10}) \times 3.0616 = 6.1232\times10^{-10}\,\text{m}
\]

\textbf{(b) Densidade de probabilidade:}
\[
P(x,t) = \frac{1}{\sqrt{2\pi}\sigma_t} e^{-\frac{(x - vt)^2}{2\sigma_t^2}}
\]

\textbf{(c) Posição central:}
\[
x_c = vt = (5.7847\times10^5)(2.0\times10^{-15}) = 1.1569\times10^{-9}\,\text{m}
\]

\section*{5. Interpretação física}

\textbf{(a) Largura do pacote:}  
A largura $\sigma_t$ aumenta com o tempo devido à dispersão quântica. Isso ocorre porque diferentes componentes de momento do pacote se propagam com velocidades ligeiramente diferentes, resultando em um espalhamento gradual da função de onda.

\textbf{(b) Posição média:}  
A posição média $\langle x \rangle = x_c(t)$ se move com velocidade constante $v = p_0/m_e$, ou seja, o centro do pacote segue a trajetória clássica de uma partícula livre.

\section*{Conclusão}

Com o passar do tempo, o pacote gaussiano se desloca e se espalha simultaneamente.  
O centro se move como uma partícula clássica livre, enquanto a incerteza em posição aumenta devido à dispersão quântica.

\end{document}
