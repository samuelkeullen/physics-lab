\documentclass[12pt,a4paper]{article}

\usepackage[utf8]{inputenc}
\usepackage{amsmath, amssymb}
\usepackage{geometry}
\usepackage{siunitx}

\geometry{margin=2.5cm}

\title{Exercícios de Correção de 1-loop em Campo Escalar $\phi^4$}
\author{Samuel Keullen Sales}
\date{\today}

\begin{document}

\maketitle

\section*{Exercício 1: Correção de 1-loop de self-energy (tadpole)}

\subsection*{Dados}
\begin{itemize}
    \item Modelo: $\phi^4$ escalar real, unidades naturais $\hbar = c = 1$.
    \item Massa: $m = \SI{2}{\electronvolt}$
    \item Acoplamento: $\lambda = 0.05$
    \item Escala de renormalização: $\mu = \SI{2}{\electronvolt}$
\end{itemize}

\subsection*{Correção de 1-loop (dimensional regularization, esquema MS)}
\[
\Sigma_\text{finite} = \frac{\lambda m^2}{32\pi^2} \left(1 + \ln \frac{\mu^2}{m^2} \right)
\]

\subsection*{Cálculos passo a passo}
\begin{align*}
\text{Primeira parte: } & \lambda m^2 = 0.05 \times 2^2 = 0.2 \\
\text{Segunda parte: } & 32\pi^2 \approx 315.82 \\
\text{Terceira parte: } & 1 + \ln \frac{\mu^2}{m^2} = 1 + \ln \frac{2^2}{2^2} = 1 + \ln 1 = 1 \\
\text{Resultado final: } & \Sigma_\text{finite} = \frac{0.2}{315.82} \approx 0.000633~\text{eV} \\
\text{Conversão para } J^2: & (1.602\times 10^{-19})^2 \times 0.000633 \approx 1.624\times 10^{-41}~\text{J}^2
\end{align*}

\subsection*{Interpretação física}
$\Sigma_\text{finite}$ representa a correção finita de 1-loop para a massa do campo escalar. Ela garante que o parâmetro físico $m_\text{phys}$ seja mensurável, evitando divergências artificiais do cálculo. No contexto da QFT, é a parte da autoenergia que renormaliza a massa, tornando o sistema consistente e previsível.

\newpage
\section*{Exercício 2: Correção de 1-loop de self-energy (bubble) para propagador externo}

\subsection*{Dados}
\begin{itemize}
    \item Modelo: $\phi^4$ escalar real, $\hbar = c = 1$
    \item Massa: $m = \SI{1.5}{\electronvolt}$
    \item Acoplamento: $\lambda = 0.1$
    \item Momento externo: $p = \SI{1}{\electronvolt}$
    \item Escala de renormalização: $\mu = \SI{1}{\electronvolt}$
\end{itemize}

\subsection*{Fórmula}
\[
\Sigma_\text{finite}(p^2) = \frac{\lambda}{32\pi^2} \left[ m^2 \left(1 + \ln \frac{\mu^2}{m^2} + f(p^2, m^2) \right) \right]
\]
Consideraremos $f(p^2, m^2) = 0$ para simplificação.

\subsection*{Cálculos passo a passo}
\begin{align*}
\ln \frac{\mu^2}{m^2} &= \ln \frac{1^2}{1.5^2} = \ln 0.4444 \approx -0.8111 \\
\lambda m^2 &= 0.1 \times (1.5)^2 = 0.1 \times 2.25 = 0.225 \\
\text{Multiplicando pelo fator } (1 + \ln \frac{\mu^2}{m^2}) &: 0.225 \times (1 - 0.8111) = 0.225 \times 0.1889 \approx 0.0425 \\
32\pi^2 &\approx 315.827 \\
\Sigma_\text{finite} &= \frac{0.0425}{315.827} \approx 0.000134~(\SI{1.34e-4}{\electronvolt}) \\
\text{Conversão para } J^2 &: 1.34\times 10^{-4} \times (1.602\times 10^{-19})^2 \approx 3.438\times 10^{-42}~\text{J}^2
\end{align*}

\subsection*{Interpretação física}
$\Sigma_\text{finite}$ ajusta a massa efetiva do propagador do campo, garantindo consistência da teoria e evitando divergências artificiais.

\newpage
\section*{Exercício 3: Beta Function e running de acoplamento}

\subsection*{Dados}
\begin{itemize}
    \item Modelo: $\phi^4$ escalar real
    \item Condição inicial: $\lambda(\mu_0) = 0.1$
    \item Escala inicial: $\mu_0 = \SI{1}{\electronvolt}$
    \item Escala final: $\mu = \SI{1e3}{\electronvolt}$
\end{itemize}

\subsection*{Beta function 1-loop}
\[
\beta(\lambda) = \mu \frac{d\lambda}{d\mu} = \frac{3 \lambda^2}{16\pi^2}
\]

\subsection*{Solução da equação RG (separável)}
\[
\lambda(\mu) = \frac{\lambda(\mu_0)}{1 - \frac{3 \lambda(\mu_0)}{16\pi^2} \ln \frac{\mu}{\mu_0}}
\]

\subsection*{Cálculos passo a passo}
\begin{align*}
\frac{3\lambda(\mu_0)}{16\pi^2} &= \frac{3 \times 0.1}{157.913} \approx 0.00189 \\
\ln \frac{\mu}{\mu_0} &= \ln \frac{1000}{1} = \ln 1000 \approx 6.907 \\
0.00189 \times 6.907 &\approx 0.0130 \\
1 - 0.0130 &\approx 0.987 \\
\lambda(\mu) &= \frac{0.1}{0.987} \approx 0.1013
\end{align*}

\subsection*{Interpretação física}
O running mostra que o acoplamento $\lambda$ cresce levemente com a escala de energia, refletindo que a intensidade da interação depende do momento observado e enfatizando a necessidade de renormalização.

\end{document}
