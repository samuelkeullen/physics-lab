\documentclass[12pt,a4paper]{article}
\usepackage[utf8]{inputenc}
\usepackage{amsmath, amssymb, physics}
\usepackage{geometry}
\usepackage{siunitx}
\geometry{margin=2.5cm}

\title{EXERCÍCIO DE MECÂNICA QUÂNTICA \\ Spin e Partículas Relativísticas}
\author{Samuel Keullen Sales}
\date{\today}

\begin{document}

\maketitle

\section*{1. História e descobridores}

\textbf{Spin:}  
- Conceito introduzido para explicar o momento magnético intrínseco dos elétrons.  
- Descoberto empiricamente por Stern e Gerlach (1922).  
- Formalizado teoricamente por Pauli (1927) através das matrizes de Pauli.

\textbf{Equação de Klein--Gordon:}  
- Desenvolvida por Oskar Klein e Walter Gordon (1926--1927).  
- Primeira equação relativística para partículas de spin 0 (bósons).

\textbf{Equação de Dirac:}  
- Desenvolvida por Paul Dirac (1928).  
- Descreve partículas de spin \(1/2\) e prevê a existência da antimatéria (pósitron).

\section*{2. Conceito de Spin}

- Spin é um momento angular intrínseco, com magnitude:
\[
S^2 = s(s+1)\hbar^2,
\qquad S_z = m_s \hbar,\quad m_s=-s,\dots,s.
\]
- Férmions: spin semi-inteiro (estatística de Fermi--Dirac).  
- Bósons: spin inteiro (estatística de Bose--Einstein).

\section*{3. Equações relativísticas (forma formal e destrinchada)}

\subsection*{3.1 Equação de Klein--Gordon (spin 0)}

\textbf{Forma covariante:}
\[
\left(\frac{1}{c^2}\frac{\partial^2}{\partial t^2} - \nabla^2 + \frac{m^2 c^2}{\hbar^2}\right)\phi(\mathbf{r},t) = 0.
\]

\textbf{Derivação (passo-a-passo):}
\begin{align*}
E^2 &= p^2 c^2 + m^2 c^4, \\
E &\to i\hbar\partial_t, \qquad \mathbf{p}\to -i\hbar\nabla, \\
(i\hbar\partial_t)^2 \phi &= (-i\hbar c \nabla)^2 \phi + m^2 c^4 \phi, \\
-\hbar^2 \partial_t^2 \phi &= -\hbar^2 c^2 \nabla^2 \phi + m^2 c^4 \phi, \\
\Rightarrow \left(\frac{1}{c^2}\partial_t^2 - \nabla^2 + \frac{m^2 c^2}{\hbar^2}\right)\phi &= 0.
\end{align*}

\subsection*{3.2 Equação de Dirac (spin \(1/2\))}

\textbf{Forma covariante:}
\[
\big( i\hbar \gamma^\mu \partial_\mu - mc \big)\psi = 0,
\]
onde \(\psi\) é um espinor de 4 componentes e as \(\gamma^\mu\) satisfazem \(\{\gamma^\mu,\gamma^\nu\}=2\eta^{\mu\nu}\).

\textbf{Construção (intuição resumida):}
\begin{itemize}
  \item Procuramos uma equação linear em \(E\) e \(\mathbf{p}\).
  \item Propomos \(E = \boldsymbol{\alpha}\cdot \mathbf{p}c + \beta mc^2\) com \(\alpha_i,\beta\) matrizes que anticomutam; isso leva às matrizes \(\gamma^\mu\).
  \item Soluções de ondas planas fornecem espinores \(u(\mathbf{p})\) (partículas) e \(v(\mathbf{p})\) (antipartículas).
\end{itemize}

\section*{4. Matrizes de Pauli e operadores de spin}

Matrizes de Pauli:
\[
\sigma_x = \begin{pmatrix}0 & 1\\ 1 & 0\end{pmatrix},\quad
\sigma_y = \begin{pmatrix}0 & -i\\ i & 0\end{pmatrix},\quad
\sigma_z = \begin{pmatrix}1 & 0\\ 0 & -1\end{pmatrix}.
\]
Operador de spin (para spin \(1/2\)):
\[
\hat{S}_i = \frac{\hbar}{2}\sigma_i.
\]

\section*{5. Aplicações (resumo)}

\begin{itemize}
  \item Spin: MRI, ressonância eletrônica, qubits (computação quântica), espectroscopia.
  \item Klein--Gordon: campos escalares em teoria quântica de campos, cosmologia primitiva.
  \item Dirac: elétrons, antipartículas, física de partículas, grafeno (análogos relativísticos).
\end{itemize}

\newpage
\section*{6. Exemplos numéricos resolvidos (após a base teórica)}

% -------------------------------------------------------------------------
% 1. EQUAÇÃO DE DIRAC: exemplos A e B
% -------------------------------------------------------------------------
\subsection*{1. Equação de Dirac (exemplos numéricos)}

\subsubsection*{Exemplo 1.A — Energia de repouso do elétron \(mc^2\)}

Constantes usadas:
\[
m_e = 9.10938356\times 10^{-31}\ \mathrm{kg},\qquad
c = 2.99792458\times 10^8\ \mathrm{m/s},
\]
e
\[
1\ \mathrm{eV} = 1.602176634\times 10^{-19}\ \mathrm{J}.
\]

Cálculo (passo a passo, dígito a dígito):

\begin{align*}
c^2 &= (2.99792458\times 10^{8}\ \mathrm{m/s})^2
= 8.9875517873681764\times 10^{16}\ \mathrm{m^2/s^2}, \\[6pt]
mc^2 &= m_e c^2
= (9.10938356\times 10^{-31}\ \mathrm{kg}) \times (8.9875517873681764\times 10^{16}\ \mathrm{m^2/s^2}) \\
&= 8.187105649650028\times 10^{-14}\ \mathrm{J}.
\end{align*}

Converter para elétron-volts:
\[
mc^2 = \frac{8.187105649650028\times 10^{-14}\ \mathrm{J}}{1.602176634\times 10^{-19}\ \mathrm{J/eV}}
= 510\,998.9420585963\ \mathrm{eV} \approx 0.510999\ \mathrm{MeV}.
\]

\emph{Conclusão:} energia de repouso do elétron \(mc^2 \approx 8.1871\times10^{-14}\ \mathrm{J}\approx 0.511\ \mathrm{MeV}.\)

\subsubsection*{Exemplo 1.B — Energia relativística para um momento dado (elétron)}

Relação usada:
\[
E = \sqrt{(pc)^2 + (mc^2)^2}.
\]

Considere
\[
p = 1.0\times 10^{-22}\ \mathrm{kg\cdot m/s}.
\]

Calcule \(pc\) (passo a passo):
\begin{align*}
pc &= p\,c
= (1.0\times 10^{-22}\ \mathrm{kg\,m/s})\times(2.99792458\times 10^8\ \mathrm{m/s}) \\
   &= 2.99792458\times 10^{-14}\ \mathrm{J}.
\end{align*}

Observação: a unidade é consistente, pois \(\mathrm{kg\,m/s}\times\mathrm{m/s}=\mathrm{kg\,m^2/s^2}=\mathrm{J}\).

Agora calcule os quadrados:
\[
(pc)^2 = (2.99792458\times 10^{-14})^2 = 8.987551787\times 10^{-28}\ \mathrm{J}^2,
\]
e recordando \(mc^2 = 8.187105649650028\times 10^{-14}\ \mathrm{J}\),
\[
(mc^2)^2 = (8.187105649650028\times 10^{-14})^2 = 6.703065000\times 10^{-27}\ \mathrm{J}^2.
\]

Some os termos:
\[
(pc)^2 + (mc^2)^2 \approx 8.987551787\times 10^{-28} + 6.703065000\times 10^{-27}
= 7.6018201787\times 10^{-27}\ \mathrm{J}^2.
\]

Tome a raiz:
\[
E = \sqrt{7.6018201787\times 10^{-27}}\ \mathrm{J}
= 8.718729879168156\times 10^{-14}\ \mathrm{J}.
\]

Converter para eV:
\[
E = \frac{8.718729879168156\times 10^{-14}\ \mathrm{J}}{1.602176634\times 10^{-19}\ \mathrm{J/eV}}
\approx 544\,180.3165860024\ \mathrm{eV} \approx 0.54418\ \mathrm{MeV}.
\]

Energia cinética relativística \(K = E - mc^2\):
\begin{align*}
K &= 8.718729879168156\times 10^{-14}\ \mathrm{J} - 8.187105649650028\times 10^{-14}\ \mathrm{J} \\
  &= 5.316242295181279\times 10^{-15}\ \mathrm{J} \\
  &\approx \frac{5.316242295181279\times 10^{-15}\ \mathrm{J}}{1.602176634\times 10^{-19}\ \mathrm{J/eV}}
  \approx 33\,181.374527406064\ \mathrm{eV} \\
  &\approx 33.18\ \mathrm{keV}.
\end{align*}

\emph{Conclusão:} Para \(p=1\times10^{-22}\ \mathrm{kg\,m/s}\) o elétron tem energia total \(E\approx 0.54418\ \mathrm{MeV}\) e energia cinética \(K\approx 33.18\ \mathrm{keV}.\)

% -------------------------------------------------------------------------
% 2. EQUAÇÃO DE KLEIN-GORDON: exemplos A e B
% -------------------------------------------------------------------------
\subsection*{2. Equação de Klein--Gordon (exemplos numéricos)}

\subsubsection*{Exemplo 2.A — Solução de onda plana e relação de dispersão (formal)}

Procuramos soluções do tipo onda plana:
\[
\phi(\mathbf{r},t) = e^{i(\mathbf{k}\cdot\mathbf{r}-\omega t)}.
\]
Substituindo na equação de Klein--Gordon obtém-se a relação de dispersão:
\[
\frac{\omega^2}{c^2} - k^2 = \frac{m^2 c^2}{\hbar^2}
\quad\Longrightarrow\quad
\omega(k) = \sqrt{(ck)^2 + \left(\frac{mc^2}{\hbar}\right)^{\!2}}.
\]
Energia associada à onda (por quantum): \(E=\hbar\omega\), que leva à relação usual
\[
E^2 = (pc)^2 + (mc^2)^2,
\quad\text{com } p=\hbar k.
\]

\subsubsection*{Exemplo 2.B — Exemplo numérico aplicado (píon e modo com \(k=10^{12}\ \mathrm{m^{-1}}\))}

Aqui usamos um bóson escalar de massa semelhante ao píon (exemplo típico em física de partículas).  
Dados e constantes:
\[
\hbar = 1.054571817\times 10^{-34}\ \mathrm{J\,s},\qquad
c = 2.99792458\times 10^{8}\ \mathrm{m/s},
\]
e massa do píon (convertida a partir de \(139.57\ \mathrm{MeV}/c^2\)):
\[
m_\pi = \frac{139.57\times 10^6\ \mathrm{eV}\times 1.602176634\times 10^{-19}\ \mathrm{J/eV}}{c^2}
\approx 2.488061244016057\times 10^{-28}\ \mathrm{kg}.
\]

Escolhemos um modo com número de onda
\[
k = 1.0\times 10^{12}\ \mathrm{m^{-1}}.
\]

Cálculo de \(\omega\) usando a relação de dispersão:
\[
\omega = \sqrt{(ck)^2 + \left(\frac{m_\pi c^2}{\hbar}\right)^{\!2}}.
\]

Calcule os termos (passo a passo):
\begin{align*}
ck &= (2.99792458\times 10^{8}\ \mathrm{m/s})\times(1.0\times 10^{12}\ \mathrm{m^{-1}})
= 2.99792458\times 10^{20}\ \mathrm{s^{-1}}, \\[6pt]
\frac{m_\pi c^2}{\hbar}
&= \frac{(2.488061244016057\times 10^{-28}\ \mathrm{kg})\times (2.99792458\times 10^{8}\ \mathrm{m/s})^2}{1.054571817\times 10^{-34}\ \mathrm{J\,s}} \\
&= \frac{(2.488061244016057\times 10^{-28})\times 8.9875517873681764\times 10^{16}\ \mathrm{J}}{1.054571817\times 10^{-34}\ \mathrm{J\,s}} \\
&= \frac{2.2361601629889656\times 10^{-11}\ \mathrm{J}}{1.054571817\times 10^{-34}\ \mathrm{J\,s}}
= 2.12044369756656\times 10^{23}\ \mathrm{s^{-1}}.
\end{align*}

Agora:
\[
(ck)^2 = (2.99792458\times 10^{20})^2 = 8.987551787\times 10^{40}\ \mathrm{s^{-2}},
\]
\[
\left(\frac{m_\pi c^2}{\hbar}\right)^2 = (2.12044369756656\times 10^{23})^2 = 4.495282\!\times 10^{46}\ \mathrm{s^{-2}}.
\]

Somando (o termo de massa domina aqui):
\[
\omega^2 = 8.987551787\times 10^{40} + 4.495282\times 10^{46} \approx 4.495282\times 10^{46}\ \mathrm{s^{-2}}.
\]

Portanto:
\[
\omega = \sqrt{4.495282\times 10^{46}}\ \mathrm{s^{-1}} \approx 2.12044369756656\times 10^{23}\ \mathrm{s^{-1}}.
\]

Energia associada \(E=\hbar\omega\):
\[
E = (1.054571817\times 10^{-34}\ \mathrm{J\,s})\times (2.12044369756656\times 10^{23}\ \mathrm{s^{-1}})
= 2.2361601629889656\times 10^{-11}\ \mathrm{J}.
\]

Converter \(E\) para eV:
\[
E = \frac{2.2361601629889656\times 10^{-11}\ \mathrm{J}}{1.602176634\times 10^{-19}\ \mathrm{J/eV}}
\approx 139\,570\,139.49243286\ \mathrm{eV} \approx 139.57\ \mathrm{MeV}.
\]

\emph{Conclusão (KG):} para \(k=10^{12}\ \mathrm{m^{-1}}\) o modo corresponde a uma energia \(E\) muito próxima do repouso do píon (\(\sim139.57\ \mathrm{MeV}\)), porque o termo de massa \((mc^2/\hbar)\) domina sobre \(ck\) nesta escolha de \(k\). Assim o comportamento é de partícula massiva com pequena correção de dispersão.

\section*{7. Exercícios sugeridos (para fixação)}

\begin{enumerate}
  \item Mostrar que a solução de onda plana \(\phi(\mathbf{r},t)=\exp[i(\mathbf{k}\cdot\mathbf{r}-\omega t)]\) satisfaz a equação de Klein--Gordon e obter a relação de dispersão \(\omega(k)\).
  \item Calcular a energia \(E\) (em eV) para um elétron com momento \(p = 1\times10^{-23}\ \mathrm{kg\,m/s}\), seguindo o mesmo procedimento do Exemplo 1.B.
  \item Para um espinor de Dirac livre com momento \(\mathbf p\) ao longo do eixo \(z\), construir explicitamente os espinores de energia positiva \(u(\mathbf p)\) para spin up e spin down e verificar sua normalização.
  \item Repetir o Exemplo 2.B com \(k=10^{14}\ \mathrm{m^{-1}}\) e comparar quão importante se torna o termo \(ck\) em relação a \(mc^2/\hbar\).
\end{enumerate}

\end{document}
