\documentclass[12pt,a4paper]{article}
\usepackage{amsmath, amssymb}
\usepackage{geometry}
\geometry{margin=1in}

\title{Introdução a Lagrange e Euler-Lagrange}
\author{Samuel Keullen Sales}
\date{\today}

\begin{document}

\maketitle

\section{Objetivo}
Este documento tem como objetivo apresentar de forma intuitiva e detalhada os conceitos de Lagrange e Euler-Lagrange, preparando para estudos em física avançada, como mecânica clássica, relatividade e mecânica quântica.

\section{A Lagrangiana}

A \textbf{Lagrangiana} de um sistema é definida como:

\begin{equation}
L = T - V
\end{equation}

onde:

\begin{itemize}
    \item $T$ é a \textbf{energia cinética}, ou seja, a energia associada ao movimento. 
    É aquela energia que mede \textit{"quanto o sistema está se movendo"}.
    \[
    T = \frac{1}{2} m \dot{q}^2 \quad \text{ou} \quad T = \frac{1}{2} I \dot{\theta}^2 \text{ para rotações}
    \]
    \textbf{Intuição:} AHHH! Energia cinética em física avançada é basicamente isso: medir a liberdade e inércia do sistema.

    \item $V$ é a \textbf{energia potencial}, relacionada às forças conservativas que limitam ou restringem o movimento:
    \[
    V = m g h \quad \text{(gravidade)}, \qquad V = \frac{1}{2} k x^2 \quad \text{(mola)}
    \]
    \textbf{Intuição:} Essa é a energia que “puxa” o sistema de volta, regula seu movimento e define a tendência de equilíbrio.
\end{itemize}

Portanto, a Lagrangiana mede o \textit{conflito dinâmico} entre movimento livre e restrição. O caminho físico real é aquele que torna a ação
\begin{equation}
S = \int L \, dt
\end{equation}
\textbf{estacionária} (\(\delta S = 0\)).

\subsection{Resumo Intuitivo}
\begin{itemize}
    \item $T$: quer manter o movimento, representa inércia e liberdade.
    \item $V$: quer trazer de volta, representa restrição e equilíbrio.
    \item $L = T - V$: mede o conflito entre liberdade e restrição. O sistema escolhe naturalmente o caminho de mínima ação.
\end{itemize}

\section{Equações de Euler-Lagrange}

A partir da Lagrangiana, as equações de movimento de um sistema podem ser obtidas por:

\begin{equation}
\frac{d}{dt}\left(\frac{\partial L}{\partial \dot{q}_i}\right) - \frac{\partial L}{\partial q_i} = 0
\end{equation}

onde \(q_i\) são as coordenadas generalizadas do sistema.

\subsection{Exemplo 1: Pêndulo Simples}

\begin{itemize}
    \item Comprimento: $l = 0.5 \, m$
    \item Ângulo inicial: $\theta_0 = 10^\circ$
    \item Gravidade: $g = 9.81 \, m/s^2$
\end{itemize}

A Lagrangiana é:
\[
L = T - V = \frac{1}{2} m l^2 \dot{\theta}^2 - m g l (1 - \cos\theta)
\]

Aplicando Euler-Lagrange:
\[
\frac{d}{dt} \left(\frac{\partial L}{\partial \dot{\theta}}\right) - \frac{\partial L}{\partial \theta} = 0
\]
\[
m l^2 \ddot{\theta} + m g l \sin\theta = 0
\]

\subsection{Exemplo 2: Massa em Mola}

\begin{itemize}
    \item Massa: $m$
    \item Constante elástica: $k$
\end{itemize}

Lagrangiana:
\[
L = \frac{1}{2} m \dot{x}^2 - \frac{1}{2} k x^2
\]

Euler-Lagrange:
\[
m \ddot{x} + k x = 0
\]

\section{Perguntas e respostas rápidas}

\textbf{1. É normal aplicar Lagrange apenas com exemplos no começo?}\\
Sim. Inicialmente, o foco é entender o procedimento e interpretar fisicamente as equações. Com o tempo, ao estudar Noether e sistemas mais complexos, sua intuição se solidifica.

\textbf{2. O pêndulo é um oscilador harmônico?}\\
Aproximadamente, para pequenos ângulos, sim. A aproximação linear (\(\sin\theta \approx \theta\)) faz com que se comporte como um oscilador harmônico.

\textbf{3. Preciso estudar partículas antes de Noether?}\\
Não é estritamente necessário. Sistemas como pêndulos ou massas em molas são suficientes para desenvolver intuição inicial.

\end{document}
