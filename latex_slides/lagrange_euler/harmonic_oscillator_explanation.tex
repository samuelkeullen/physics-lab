\documentclass[12pt,a4paper]{article}
\usepackage[utf8]{inputenc}
\usepackage{amsmath}
\usepackage{amsfonts}
\usepackage{amssymb}
\usepackage{geometry}
\geometry{margin=2.5cm}

\title{Entendendo Osciladores Harmônicos: Pêndulo e Mola}
\author{Samuel Keullen Sales}
\date{\today}

\begin{document}

\maketitle

\section*{Introdução}

Este documento apresenta uma explicação didática sobre o comportamento energético de dois exemplos clássicos de osciladores harmônicos:  
o pêndulo simples e o sistema massa-mola. O objetivo é compreender como energia cinética e energia potencial se alternam durante a oscilação, utilizando conceitos de Lagrangiana.

\section{Sistema Massa–Mola}

A Lagrangiana para um sistema massa–mola é dada por:
\[
L = T - V
\]
onde:
\[
T = \frac{1}{2} m \dot{x}^2, \quad V = \frac{1}{2} k x^2
\]
Assim:
\[
L(x, \dot{x}, t) = \frac{1}{2} m \dot{x}^2 - \frac{1}{2} k x^2
\]

\subsection*{Interpretação energética}

\begin{itemize}
    \item \textbf{Posição máxima:} o deslocamento \(x\) é máximo, \(\dot{x} = 0\).  
    Energia cinética \(T = 0\) e energia potencial \(V\) é máxima. A energia armazenada na mola gera a força restauradora.
    
    \item \textbf{Ponto de equilíbrio:} \(x = 0\), \(\dot{x}\) é máximo.  
    Energia cinética \(T\) é máxima e energia potencial \(V = 0\). Toda a energia está associada ao movimento.
\end{itemize}

Essa alternância constante entre \(T\) e \(V\) caracteriza o movimento harmônico simples.

\section{Pêndulo Simples}

Para pequenas amplitudes, a Lagrangiana do pêndulo é:
\[
L = \frac{1}{2} m l^2 \dot{\theta}^2 - m g l (1 - \cos\theta)
\]
onde:
\[
\theta \ \text{é o ângulo de deslocamento}, \quad l \ \text{comprimento do pêndulo}.
\]

\subsection*{Interpretação energética}

\begin{itemize}
    \item \textbf{Posição máxima:} \(\theta\) máximo, \(\dot{\theta} = 0\).  
    Energia cinética \(T = 0\), energia potencial \(V\) máxima.  
    A gravidade fornece a força restauradora para iniciar o movimento.
    
    \item \textbf{Ponto de equilíbrio:} \(\theta = 0\), \(\dot{\theta}\) máximo.  
    Energia cinética \(T\) máxima, energia potencial \(V \approx 0\).  
    A velocidade permite que o pêndulo ultrapasse o ponto de equilíbrio e continue a oscilação.
\end{itemize}

Assim, no pêndulo, a energia alterna entre cinética e potencial, mantendo o movimento harmônico.

\section{Conclusão}

Tanto no sistema massa–mola quanto no pêndulo simples, a oscilação é caracterizada por uma troca periódica entre energia cinética e energia potencial.  
Esse balanço harmônico é a razão pela qual chamamos esses sistemas de \emph{osciladores harmônicos}, sendo que sua descrição através da Lagrangiana fornece uma poderosa ferramenta para análise.

\end{document}
