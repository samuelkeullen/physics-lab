\documentclass[12pt,a4paper]{article}
\usepackage[utf8]{inputenc}
\usepackage{amsmath, amssymb}
\usepackage{graphicx}
\usepackage{caption}
\usepackage{hyperref}

\title{Onda Linear e Superposição}
\author{Samuel Keullen Sales}
\date{\today}

\begin{document}

\maketitle

\section{Introdução}
Ondas aparecem em diversos contextos da física, como som, luz e vibrações em cordas. 
Este documento apresenta de forma intuitiva o conceito de \textbf{onda linear} e \textbf{superposição} de ondas, com fórmulas matemáticas e um exemplo prático.

\section{Onda Linear}
Uma onda linear simples pode ser descrita por:

\begin{equation}
y(x,t) = A \sin(kx - \omega t + \phi)
\end{equation}

\noindent onde:
\begin{itemize}
    \item $A$ → amplitude (tamanho da onda),
    \item $k$ → número de onda (relaciona-se ao comprimento de onda $\lambda$ por $k = \frac{2\pi}{\lambda}$),
    \item $\omega$ → frequência angular (relaciona-se ao período $T$ por $\omega = \frac{2\pi}{T}$),
    \item $\phi$ → fase inicial,
    \item $x$ → posição,
    \item $t$ → tempo.
\end{itemize}

\subsection*{Intuição}
Podemos desmembrar a fórmula como:
\[
y(x,t) = A \cdot \sin(kx - \omega t + \phi)
\]
- $kx$ representa a posição ao longo do espaço,  
- $\omega t$ representa a evolução temporal da onda,  
- $\phi$ ajusta a fase inicial da oscilação.  

\section{Superposição de Ondas}
Quando duas ou mais ondas se encontram, a \textbf{superposição} determina que a onda resultante é a soma das ondas individuais:

\begin{equation}
y_{\text{total}}(x,t) = A_1 \sin(k_1 x - \omega_1 t + \phi_1) + A_2 \sin(k_2 x - \omega_2 t + \phi_2)
\end{equation}

O esquema é o mesmo da onda linear, porém agora consideramos a contribuição de múltiplas ondas.

\subsection*{Exemplo Prático}
Considerando:
\[
\text{onda 1: } A_1 = 2, k_1 = 1, \omega_1 = 3, \phi_1 = 0
\]
\[
\text{onda 2: } A_2 = 1, k_2 = 1, \omega_2 = 3, \phi_2 = \frac{\pi}{2}
\]

Queremos calcular $y_{\text{total}}$ em $x=0.5$ e $t=1$.

\subsubsection*{Passo 1: Argumentos das funções seno}
\[
\text{onda 1: } k_1 x - \omega_1 t + \phi_1 = 1 \cdot 0.5 - 3 \cdot 1 + 0 = -2.5
\]
\[
\text{onda 2: } k_2 x - \omega_2 t + \phi_2 = 1 \cdot 0.5 - 3 \cdot 1 + \frac{\pi}{2} \approx -0.9292
\]

\subsubsection*{Passo 2: Valores das ondas}
\[
y_1 = A_1 \sin(-2.5) \approx 2 \cdot (-0.598) = -1.196
\]
\[
y_2 = A_2 \sin(-0.9292) \approx 1 \cdot (-0.801) = -0.801
\]

\subsubsection*{Passo 3: Superposição}
\[
y_{\text{total}} = y_1 + y_2 \approx -1.196 + (-0.801) = -1.997
\]

\section{Conclusão e Relações com Mecânica Analítica}
A análise de ondas lineares e sua superposição está relacionada com princípios fundamentais da física teórica:

\begin{itemize}
    \item \textbf{Lagrange}: A onda pode ser derivada de uma função Lagrangiana, considerando energia cinética e potencial do sistema.
    \item \textbf{Euler}: As equações de Euler-Lagrange fornecem a equação de movimento para campos de onda contínuos.
    \item \textbf{Noether}: A simetria do sistema (por exemplo, invariância temporal ou espacial) leva a leis de conservação, como energia ou momento.
\end{itemize}

Portanto, mesmo conceitos aparentemente simples como ondas e superposição possuem uma conexão profunda com a formulação teórica da mecânica clássica e da física de campos.

\end{document}
