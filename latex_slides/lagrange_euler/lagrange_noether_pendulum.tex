\documentclass[12pt]{article}
\usepackage[utf8]{inputenc}
\usepackage{amsmath, amssymb, amsfonts}
\usepackage{geometry}
\geometry{a4paper, margin=2.5cm}

\title{Exemplo Prático: Pêndulo Simples, Euler-Lagrange e Noether}
\author{Samuel Keullen Sales}
\date{\today}

\begin{document}

\maketitle

\section*{1. Sistema: Pêndulo Simples}

Um pêndulo simples consiste em uma massa $m$ suspensa por um fio de comprimento $l$, oscilando sob a gravidade $g$.  
A coordenada generalizada é o ângulo $\theta(t)$ que a corda forma com a vertical.

\subsection*{1.1 Energia Cinética}
\[
T = \frac{1}{2} m (l \dot{\theta})^2 = \frac{1}{2} m l^2 \dot{\theta}^2
\]

\subsection*{1.2 Energia Potencial}
Tomando o ponto mais baixo como zero de energia potencial:
\[
V = m g l (1 - \cos\theta)
\]

\subsection*{1.3 Lagrangiana}
\[
L = T - V = \frac{1}{2} m l^2 \dot{\theta}^2 - m g l (1 - \cos\theta)
\]

---

\section*{2. Ação e discretização}

A ação ao longo do intervalo $[t_1, t_2]$:
\[
S[\theta(t)] = \int_{t_1}^{t_2} L(\theta, \dot{\theta}) \, dt
\]

Discretizando para cálculos aproximados com $n$ passos de duração $\Delta t$:
\[
S \approx \sum_{i=1}^{n} L_i \Delta t
\]

\textbf{Intuição:} cada $L_i$ representa o "momento instantâneo do sistema" naquele passo de tempo, multiplicado pelo intervalo $\Delta t$.  

---

\section*{3. Equação de Euler-Lagrange}

Condição de ação estacionária:
\[
\frac{d}{dt} \frac{\partial L}{\partial \dot{\theta}} - \frac{\partial L}{\partial \theta} = 0
\]

\subsection*{3.1 Derivadas simbólicas}
\[
\frac{\partial L}{\partial \dot{\theta}} = m l^2 \dot{\theta}, \quad 
\frac{d}{dt} \frac{\partial L}{\partial \dot{\theta}} = m l^2 \ddot{\theta}
\]

\[
\frac{\partial L}{\partial \theta} = - m g l \sin\theta
\]

\subsection*{3.2 Equação de movimento}
\[
m l^2 \ddot{\theta} + m g l \sin\theta = 0
\quad \Rightarrow \quad
\ddot{\theta} + \frac{g}{l} \sin\theta = 0
\]

---

\section*{4. Aplicação de Noether (identificando simetria)}

\subsection*{4.1 Transformações infinitesimais}
Para descobrir se alguma quantidade física se conserva:

\begin{enumerate}
    \item Testamos uma pequena mudança na coordenada ou no tempo:
\[
    \theta(t) \to \theta(t) + \epsilon, \quad t \to t + \delta t
\]
    
    \item Calculamos a Lagrangiana perturbada:
\[
    L_{\text{perturbada}} = L(\theta+\epsilon, \dot{\theta}+\dot{\epsilon}, t+\delta t)
\]
    
    \item Calculamos a variação da Lagrangiana:
\[
    \delta L = L_{\text{perturbada}} - L
\]
    
    \item Se $\delta L = 0$ (ou não depende explicitamente do tempo), há simetria e, pelo Teorema de Noether, existe uma quantidade conservada:
\[
    Q = \frac{\partial L}{\partial \dot{\theta}} \delta \theta
\]
\end{enumerate}

\subsection*{4.2 Passo simbólico}
Para uma pequena mudança $\delta t$:
\[
\theta(t) \to \theta(t+\delta t) \approx \theta(t) + \dot{\theta}\, \delta t
\]

\[
L(\theta, \dot{\theta}, t) \to L(\theta + \dot{\theta} \delta t, \dot{\theta} + \ddot{\theta} \delta t, t+\delta t)
\]

Expandindo em série de Taylor e descartando ordens superiores:
\[
\delta L \approx \frac{\partial L}{\partial \theta} \dot{\theta}\delta t + \frac{\partial L}{\partial \dot{\theta}} \ddot{\theta} \delta t + \frac{\partial L}{\partial t}\delta t
= \left(\frac{d}{dt} \frac{\partial L}{\partial \dot{\theta}} - \frac{\partial L}{\partial \theta} + \frac{\partial L}{\partial t}\right) \delta t
\]

Como $\frac{d}{dt} \frac{\partial L}{\partial \dot{\theta}} - \frac{\partial L}{\partial \theta} = 0$ pela Euler-Lagrange, então:
\[
\delta L = \frac{\partial L}{\partial t} \delta t
\]

\textbf{Conclusão simbólica:} se $L$ não depende explicitamente do tempo, $\delta L = 0$ e a energia é conservada.

---

\section*{5. Cálculos numéricos}

Valores usados:
\[
m = 1 \text{ kg},\quad l = 1 \text{ m},\quad g = 9.8 \text{ m/s²},\quad \theta = 30^\circ = \pi/6, \quad \dot{\theta} = 1 \text{ rad/s}
\]

\subsection*{5.1 Energia Cinética}
\[
T = \frac{1}{2} m l^2 \dot{\theta}^2 = 0.5 \text{ J}
\]

\subsection*{5.2 Energia Potencial}
\[
V = m g l (1 - \cos\theta) = 1 \cdot 9.8 \cdot 1 \cdot (1 - \cos(\pi/6)) 
\approx 1 \cdot 9.8 \cdot (1 - 0.866) \approx 1.31 \text{ J}
\]

\subsection*{5.3 Lagrangiana e energia total}
\[
L = T - V = 0.5 - 1.31 = -0.81 \text{ J}, \quad
E = T + V = 0.5 + 1.31 = 1.81 \text{ J}
\]

\subsection*{5.4 Diferença}
\[
E - L = 1.81 - (-0.81) = 2.62 \approx 2V
\]

\textbf{Intuição numérica:} mostramos que a energia total é maior que a Lagrangiana pela quantidade $2V$, confirmando a relação simbólica $E - L = 2V$.

---

\section*{6. Passo a passo para identificar simetria}

\begin{enumerate}
    \item Perturbamos a coordenada $\theta \to \theta + \epsilon$ ou o tempo $t \to t + \delta t$.
    \item Calculamos $\delta L$ de forma simbólica ou numérica.
    \item Se $\delta L = 0$ ou não depende de $t$, temos simetria.
    \item Pelo Teorema de Noether, existe uma quantidade conservada $Q = \frac{\partial L}{\partial \dot{\theta}} \delta \theta$.
    \item Para o pêndulo simples, $\delta L$ não depende explicitamente do tempo, logo a energia total $E$ se conserva.
\end{enumerate}

---

\section*{7. Resumo do exemplo destrinchado}

\begin{itemize}
    \item Sistema: pêndulo simples, coordenada $\theta(t)$
    \item Lagrangiana: $L = \frac{1}{2} m l^2 \dot{\theta}^2 - m g l (1 - \cos\theta)$
    \item Equação de Euler-Lagrange: $\ddot{\theta} + \frac{g}{l} \sin\theta = 0$
    \item Teste de simetria: perturbar $\theta$ ou $t$, calcular $\delta L$
    \item Quantidade conservada: $Q = \frac{\partial L}{\partial \dot{\theta}} \delta \theta$ ou energia $E$ para simetria temporal
    \item Valores numéricos: $T = 0.5$ J, $V \approx 1.31$ J, $L = -0.81$ J, $E = 1.81$ J, $E-L \approx 2V \approx 2.62$ J
\end{itemize}

\end{document}
