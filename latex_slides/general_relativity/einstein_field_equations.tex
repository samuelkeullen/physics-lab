\documentclass[12pt,a4paper]{article}
\usepackage[utf8]{inputenc}
\usepackage{amsmath, amssymb}
\usepackage{geometry}
\usepackage{physics}
\usepackage{hyperref}
\usepackage{bm}

\geometry{margin=2.5cm}

\title{Equações de Campo de Einstein e Aplicações: Schwarzschild e Kerr}
\author{Samuel Keullen Sales}
\date{\today}

\begin{document}

\maketitle

\section*{Introdução e História}

As equações de campo de Einstein são a base da \textbf{Relatividade Geral}, formuladas por Albert Einstein em 1915. Elas descrevem como a matéria e energia determinam a curvatura do espaço-tempo. A origem matemática da teoria encontra raízes no trabalho de \textbf{Bernhard Riemann}, que introduziu o conceito de variedades curvas no século XIX, e de outros matemáticos como \textbf{Elwin Bruno Christoffel} e \textbf{Gregorio Ricci-Curbastro}, que desenvolveram os tensores essenciais para a formulação moderna da geometria diferencial.

A primeira solução exata das equações de Einstein em vácuo foi encontrada por \textbf{Karl Schwarzschild} em 1916, descrevendo o campo gravitacional esférico de uma massa estática. Décadas depois, \textbf{Roy Kerr} generalizou Schwarzschild para o caso de uma massa em rotação, resultando na famosa \textbf{métrica de Kerr}, fundamental para o estudo de buracos negros rotativos.

\section*{Equações de Campo de Einstein}

As equações de campo podem ser escritas na forma compacta:

\begin{equation}
G_{\mu\nu} \equiv R_{\mu\nu} - \frac{1}{2} g_{\mu\nu} R = \frac{8\pi G}{c^4} T_{\mu\nu}
\end{equation}

\noindent onde:
\begin{itemize}
    \item $G_{\mu\nu}$ é o \textbf{tensor de Einstein},
    \item $R_{\mu\nu}$ é o \textbf{tensor de Ricci},
    \item $R$ é o \textbf{escalar de Ricci},
    \item $g_{\mu\nu}$ é o \textbf{tensor métrico},
    \item $T_{\mu\nu}$ é o \textbf{tensor energia-momento},
    \item $G$ é a constante gravitacional de Newton,
    \item $c$ é a velocidade da luz.
\end{itemize}

Essa equação relaciona a geometria do espaço-tempo (lado esquerdo) com a distribuição de matéria e energia (lado direito).

\section*{Visualização das Equações}

Para estudo inicial, é útil expressar as métricas em coordenadas esféricas $(t, r, \theta, \phi)$:

\begin{equation}
ds^2 = g_{tt} dt^2 + g_{rr} dr^2 + g_{\theta\theta} d\theta^2 + g_{\phi\phi} d\phi^2 + 2 g_{t\phi} dt d\phi
\end{equation}

\noindent onde cada $g_{\mu\nu}$ depende da simetria do sistema estudado (esférica, rotacional, etc.).


% ========================================================
\section{Métrica de Schwarzschild}

\subsection*{Constantes e dados}
\begin{align*}
G &= 6.674\times10^{-11} \quad \text{(Constante gravitacional)}\\
c &= 3\times10^8 \quad \text{(Velocidade da luz)}\\
M &= 1.989\times10^{30} \quad \text{(Massa do Sol)}\\
r &= 6.96\times10^8 \quad \text{(Raio do Sol)}\\
\theta &= \pi/2, \quad \sin\theta = 1
\end{align*}

\subsection*{Valores intermediários (arredondados)}
\begin{align*}
2GM &= 2.655\times10^{20}\\
c^2 &= 9\times10^{16}\\
r^2 &= 4.844\times10^{17}\\
\frac{2GM}{c^2 r} &= 4.238\times10^{-6}\\
1 - \frac{2GM}{c^2 r} &= 0.999996\\
\left(1 - \frac{2GM}{c^2 r}\right)^{-1} &= 1.000004\\
r^2 \sin^2\theta &= 4.844\times10^{17}
\end{align*}

\subsection*{Métrica covariante $g_{\mu\nu}$}
\begin{align*}
g_{tt} &= -0.999996\\
g_{rr} &= 1.000004\\
g_{\theta\theta} &= 4.844\times10^{17}\\
g_{\phi\phi} &= 4.844\times10^{17}
\end{align*}

\[
g_{\mu\nu} =
\begin{bmatrix}
-0.999996 & 0 & 0 & 0 \\
0 & 1.000004 & 0 & 0 \\
0 & 0 & 4.844\times10^{17} & 0 \\
0 & 0 & 0 & 4.844\times10^{17}
\end{bmatrix}
\]

\subsection*{Métrica contravariante $g^{\mu\nu}$}
\begin{align*}
g^{tt} &= -1.000004\\
g^{rr} &= 0.999996\\
g^{\theta\theta} &= 2.064\times10^{-18}\\
g^{\phi\phi} &= 2.064\times10^{-18}
\end{align*}

\[
g^{\mu\nu} =
\begin{bmatrix}
-1.000004 & 0 & 0 & 0 \\
0 & 0.999996 & 0 & 0 \\
0 & 0 & 2.064\times10^{-18} & 0 \\
0 & 0 & 0 & 2.064\times10^{-18}
\end{bmatrix}
\]

\subsection*{Conexões de Christoffel não nulas}
\begin{align*}
\Gamma^r_{tt} &= -3.045\times10^{-15} &
\Gamma^r_{rr} &= 3.045\times10^{-15} \\
\Gamma^r_{\theta\theta} &= -6.96\times10^8 &
\Gamma^r_{\phi\phi} &= -6.96\times10^8 \\
\Gamma^t_{tr} &= -3.045\times10^{-15} &
\Gamma^\theta_{r\theta} &= 1.437\times10^{-9} \\
\Gamma^\phi_{r\phi} &= 1.437\times10^{-9} &
\Gamma^\theta_{\phi\phi} &= 0 \\
\Gamma^\phi_{\theta\phi} &= 0
\end{align*}

\subsection*{Tensor de Ricci, Escalar e Einstein}
\[
R_{\mu\nu} \approx 0, \quad R \approx 0, \quad G_{\mu\nu} \approx 0
\]

\subsection*{Tensor de Riemann}
\[
R^r_{\theta r \theta} \approx 4.32\times10^{-16} \, \text{m}^{-2}
\]

---

% ========================================================
\section{Métrica de Kerr (equador, $\theta = \pi/2$)}

\subsection*{Constantes e dados}
\begin{align*}
M &= 1.989\times10^{30} & G &= 6.674\times10^{-11} & c &= 3\times10^8\\
\theta &= \pi/2, \quad \sin\theta = 1 & r &= 6.96\times10^8\\
r_s &= 2GM/c^2 = 2949.91 & a &= 0.5 r_s = 1.47495\times10^3
\end{align*}

\subsection*{Definições e preliminares}
\begin{align*}
p^2 &= r^2 + a^2 \cos^2\theta = 4.844\times10^{17}\\
\Delta &= r^2 - 2 G M r/c^2 + a^2 = 4.84416\times10^{17}
\end{align*}

\subsection*{Métrica covariante $g_{\mu\nu}$ (equador)}
\begin{align*}
g_{tt} &= -0.999996\\
g_{t\phi} &= -6.251\times10^{-3}\\
g_{rr} &= 1.0\\
g_{\theta\theta} &= 4.844\times10^{17}\\
g_{\phi\phi} &= 4.84416\times10^{17}
\end{align*}

\[
g_{\mu\nu} =
\begin{bmatrix}
-0.999996 & 0 & 0 & -6.251\times10^{-3} \\
0 & 1.0 & 0 & 0 \\
0 & 0 & 4.844\times10^{17} & 0 \\
-6.251\times10^{-3} & 0 & 0 & 4.84416\times10^{17}
\end{bmatrix}
\]

\subsection*{Métrica contravariante $g^{\mu\nu}$}
\begin{align*}
g^{tt} &= -1.000004\\
g^{t\phi} &= -1.291\times10^{-20}\\
g^{rr} &= 1.0\\
g^{\theta\theta} &= 2.064\times10^{-18}\\
g^{\phi\phi} &= 2.064\times10^{-18}
\end{align*}

\[
g^{\mu\nu} =
\begin{bmatrix}
-1.000004 & 0 & 0 & -1.291\times10^{-20} \\
0 & 1.0 & 0 & 0 \\
0 & 0 & 2.064\times10^{-18} & 0 \\
-1.291\times10^{-20} & 0 & 0 & 2.064\times10^{-18}
\end{bmatrix}
\]

\subsection*{Conexões de Christoffel principais (valores arredondados)}
\begin{align*}
\Gamma^r_{tt} &= 3.045\times10^{-15} &
\Gamma^r_{t\phi} &= -4.486\times10^{-12} \\
\Gamma^r_{rr} &= -3.042\times10^{-15} &
\Gamma^r_{\theta\theta} &= -6.96\times10^8 \\
\Gamma^r_{\phi\phi} &= -6.96\times10^8 &
\Gamma^t_{tr} &= 3.045\times10^{-15} \\
\Gamma^\phi_{tr} &\approx 9.27\times10^{-30} &
\Gamma^\theta_{r\theta} &= 1.437\times10^{-9} \\
\Gamma^\phi_{r\phi} &= 1.437\times10^{-9}
\end{align*}

\subsection*{Tensor de Riemann (exemplos)}
\begin{align*}
R^r_{\theta r \theta} &\approx 0\\
R^r_{\phi r \phi} &\approx 0\\
R^t_{r t r} &\approx -9.27\times10^{-30}
\end{align*}

\subsection*{Tensor de Ricci, Escalar e Einstein}
\[
R_{\mu\nu} \approx 0, \quad R \approx 0, \quad G_{\mu\nu} \approx 0
\]

\subsection*{Observações físicas e arrasto de referência}

\begin{itemize}
\item Campo quase plano (Riemann $\sim 10^{-30}$)
\item Espaço-tempo de Kerr no equador, longe do horizonte ($r \gg r_s$)
\item Tensor de Einstein $\approx 0$ → vácuo
\item $g_{t\phi} \neq 0$ indica leve arrasto de referência (efeito relativístico)
\item Correções de rotação ($a \sim 10^3\,\text{m}$) são minúsculas comparadas ao raio solar ($\sim 10^9\,\text{m}$)
\end{itemize}

\begin{center}
\begin{tabular}{|c|c|c|}
\hline
Componente & Schwarzschild & Kerr (equador) \\
\hline
$g_{tt}$ & -0.999996 & -0.999996 \\
$g_{rr}$ & 1.000004 & 1.0 \\
$g_{\theta\theta}$ & $4.844\times10^{17}$ & $4.844\times10^{17}$ \\
$g_{\phi\phi}$ & $4.844\times10^{17}$ & $4.84416\times10^{17}$ \\
$g_{t\phi}$ & 0 & -0.006251 \\
\hline
\end{tabular}
\end{center}

A tabela ilustra claramente o efeito de arrasto de referência: apenas Kerr possui $g_{t\phi} \neq 0$.

\end{document}
