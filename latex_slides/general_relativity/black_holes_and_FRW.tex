\documentclass[12pt,a4paper]{article} \usepackage[utf8]{inputenc} \usepackage{amsmath, amssymb} \usepackage{geometry} \usepackage{hyperref} \usepackage{graphicx} \geometry{margin=1in} \title{Espaço-Tempo Extremo: Buracos Negros e Cosmologia Básica} \author{Samuel Keullen Sales} \date{\today}

\begin{document} \maketitle

\section*{1. Buracos Negros}

\subsection*{História e Descobridores} \begin{itemize} \item 1796: John Michell e Pierre-Simon Laplace propuseram "estrelas escuras", cuja gravidade impede a luz de escapar. \item 1915-1916: Einstein publica a Teoria da Relatividade Geral. \item 1916: Karl Schwarzschild encontra a solução exata para uma massa esférica não carregada e sem rotação. \item 1960s: John Wheeler populariza o termo "black hole". Roger Penrose e Stephen Hawking estudam singularidades. \end{itemize}

\subsection*{Conceitos Principais}

\subsubsection*{Horizonte de Eventos} Limite além do qual nada escapa do buraco negro. Intuição física: ponto sem retorno.

\subsubsection*{Singularidade} Ponto central de densidade infinita e curvatura extrema. Intuição física: "coração" do buraco negro.

\subsection*{Fórmulas Relevantes}

\textbf{Raio de Schwarzschild:} \begin{equation} r_s = \frac{2GM}{c^2} \end{equation} \begin{itemize} \item $G$ = constante gravitacional ($6.674 \times 10^{-11} , \mathrm{m^3 kg^{-1} s^{-2}}$) \item $M$ = massa do buraco negro \item $c$ = velocidade da luz ($3.0 \times 10^8 , \mathrm{m/s}$) \end{itemize}

\textbf{Dilatação do tempo próximo ao horizonte:} \begin{equation} d\tau = \sqrt{1 - \frac{2GM}{rc^2}} , dt \end{equation} \begin{itemize} \item $d\tau$ = tempo medido pelo observador próximo \item $dt$ = tempo medido pelo observador distante \item $r$ = distância radial do centro do buraco negro \end{itemize}

\subsection*{Exemplo de Cálculo} \textbf{Raio de Schwarzschild de um buraco negro de 10 massas solares ($M_\odot \approx 2 \times 10^{30} , \mathrm{kg}$):} \begin{equation} r_s = \frac{2 \cdot 6.674 \times 10^{-11} \cdot 10 \cdot 2 \times 10^{30}}{(3.0 \times 10^8)^2} \approx 29.7 , \mathrm{km} \end{equation}

\subsection*{Exercício Resolvido} \textbf{Tempo próprio a 3 $r_s$ de um buraco negro de 10 massas solares durante 1 hora para observador distante:} \begin{equation} d\tau = \sqrt{1 - \frac{2GM}{rc^2}} , dt = \sqrt{1 - \frac{2 \cdot 6.674 \times 10^{-11} \cdot 10 \cdot 2 \times 10^{30}}{89100 \cdot 10^3 \cdot (3 \cdot 10^8)^2}} \cdot 3600 , s \approx 3172 , s \end{equation} Resposta: ~53 minutos e 2 segundos.

\newpage \section*{2. Cosmologia Básica (FRW)}

\subsection*{História} \begin{itemize} \item 1920s: Edwin Hubble descobre que galáxias se afastam → universo em expansão. \item 1930s: Friedmann e Lemaître desenvolvem soluções para as equações de Einstein. \item 1998: Descobre-se expansão acelerada (energia escura). \end{itemize}

\subsection*{Conceitos Principais} \subsubsection*{Métrica FRW (Friedmann–Robertson–Walker)} \begin{equation} ds^2 = -c^2 dt^2 + a(t)^2 \big(d\chi^2 + \chi^2(d\theta^2 + \sin^2 \theta , d\phi^2)\big) \end{equation} \begin{itemize} \item $a(t)$ = fator de escala \item $\chi$ = coordenada comoving radial \item $\theta, \phi$ = coordenadas angulares \end{itemize}

\subsubsection*{Equações de Friedmann} \textbf{Primeira equação:} \begin{equation} \left(\frac{\dot{a}}{a}\right)^2 = \frac{8 \pi G}{3} \rho - \frac{k c^2}{a^2} + \frac{\Lambda c^2}{3} \end{equation} \textbf{Segunda equação:} \begin{equation} \frac{\ddot{a}}{a} = -\frac{4 \pi G}{3} \left(\rho + \frac{3p}{c^2}\right) + \frac{\Lambda c^2}{3} \end{equation} Legenda: \begin{itemize} \item $\rho$ = densidade de energia do universo \item $p$ = pressão média \item $k$ = curvatura espacial (0 plano, +1 fechado, -1 aberto) \item $\Lambda$ = constante cosmológica \end{itemize}

\subsection*{Exemplo de Cálculo} \textbf{Distância física radial de uma galáxia comoving $\chi = 3000 , \mathrm{Mpc}$:} \begin{equation} D(t) = a(t) \chi \end{equation} Hoje ($a_0 = 1$): $D_0 = 3000 , \mathrm{Mpc}$

Passado ($a = 0.5$): $D_\mathrm{passado} = 1500 , \mathrm{Mpc}$

\subsubsection*{Incluindo ângulos} \begin{equation} D_\mathrm{angular} = a(t) \sqrt{\chi^2 + \chi^2 (\Delta\theta^2 + \sin^2\theta , \Delta\phi^2)} \end{equation} \begin{itemize} \item $\theta = 45^\circ = 0.785$ rad, $\phi = 30^\circ = 0.524$ rad \item $\chi^2(\Delta\theta^2 + \sin^2\theta \Delta\phi^2) = 9{,}000{,}000 \cdot 0.753 \approx 6{,}777{,}000$ \item Soma com $\chi^2$ radial: $9{,}000{,}000 + 6{,}777{,}000 = 15{,}777{,}000$ \item Raiz quadrada: $D_\mathrm{angular} \approx 3972 , \mathrm{Mpc}$ \end{itemize}

\subsection*{Exercícios Resolvidos} \begin{enumerate} \item Distância física radial hoje: $D_0 = 3000 , \mathrm{Mpc}$ \item Distância física radial no passado ($a=0.5$): $D_\mathrm{passado} = 1500 , \mathrm{Mpc}$ \item Distância física 3D hoje (com ângulos): $D_\mathrm{angular} \approx 3972 , \mathrm{Mpc}$ \end{enumerate}

\end{document}

