\documentclass[a4paper,12pt]{article}
\usepackage[utf8]{inputenc}
\usepackage[brazil]{babel}
\usepackage{amsmath,amssymb,amsthm}
\usepackage{geometry}
\usepackage{graphicx}
\usepackage{setspace}
\usepackage{hyperref}

\geometry{margin=2.5cm}
\onehalfspacing

\title{\textbf{Geodésicas e Conexões no Espaço-Tempo}}
\author{Samuel Keullen Sales}
\date{\today}

\begin{document}

\maketitle

\section*{Origem da Ideia}

A noção de \textbf{geodésica} vem da geometria diferencial.  
Em superfícies como a esfera, as geodésicas são as \textbf{linhas de menor distância} entre dois pontos (ex: arcos de grandes círculos).  

Com Einstein (1915), a gravitação foi reinterpretada: não como força, mas como efeito da curvatura do espaço-tempo.  
Corpos livres seguem as \textbf{geodésicas} do espaço-tempo, não sentindo força, mas movendo-se de acordo com a geometria.

\section*{Intuição Física das Geodésicas}

- Espaço-tempo plano: um corpo sem forças move-se em linha reta com velocidade constante.  
- Espaço-tempo curvo: a “reta” é deformada, mas o corpo ainda segue o caminho mais reto possível localmente.

\begin{quote}
\textit{Uma geodésica é a generalização da linha reta para um espaço curvo.}
\end{quote}

Para partículas massivas, a geodésica \textbf{maximiza o tempo próprio}.  
Para partículas sem massa, $ds^2=0$ (geodésicas nulas).

\section*{Equação das Geodésicas}

A forma geral:

\begin{equation}
\frac{d^2 x^\mu}{d\lambda^2} + \Gamma^{\mu}_{\ \nu\rho} 
\frac{d x^\nu}{d\lambda} \frac{d x^\rho}{d\lambda} = 0
\end{equation}

\textbf{Legenda:}
\begin{itemize}
    \item $x^\mu(\lambda)$ — coordenadas do ponto na trajetória;
    \item $\lambda$ — parâmetro ao longo da curva (tempo próprio $\tau$);
    \item $\Gamma^{\mu}_{\ \nu\rho}$ — símbolos de Christoffel, definem a curvatura local;
    \item $\frac{dx^\nu}{d\lambda}$ — vetor tangente à trajetória.
\end{itemize}

\textbf{Destrinche:}
\begin{align*}
\frac{d^2 x^\mu}{d\lambda^2} &\rightarrow \text{aceleração “reta” do corpo} \\
\Gamma^{\mu}_{\ \nu\rho} \frac{dx^\nu}{d\lambda} \frac{dx^\rho}{d\lambda} &\rightarrow \text{correção devido à curvatura} \\
\text{Soma igual a zero} &\Rightarrow \text{movimento livre, sem força externa}.
\end{align*}

\section*{Conexões e Derivada Covariante}

A \textbf{conexão} define como transportar vetores de um ponto a outro em um espaço curvo.  
É formalizada via derivada covariante:

\begin{equation}
\nabla_\mu V^\nu = \partial_\mu V^\nu + \Gamma^{\nu}_{\ \mu\rho} V^\rho
\end{equation}

\textbf{Legenda:}
\begin{itemize}
    \item $\nabla_\mu$ — derivada covariante na direção $x^\mu$;
    \item $V^\nu$ — vetor a ser derivado;
    \item $\Gamma^{\nu}_{\ \mu\rho}$ — símbolos de Christoffel.
\end{itemize}

\textbf{Destrinche:}
\begin{align*}
\partial_\mu V^\nu &\rightarrow \text{variação usual do vetor} \\
\Gamma^{\nu}_{\ \mu\rho} V^\rho &\rightarrow \text{correção da curvatura} \\
\nabla_\mu V^\nu &\rightarrow \text{derivada que respeita a geometria do espaço-tempo}
\end{align*}

\subsection*{Transporte Paralelo}

Um vetor \(V^\mu\) é transportado paralelamente se:

\begin{equation}
u^\nu \nabla_\nu V^\mu = 0
\end{equation}

onde \(u^\nu = \frac{dx^\nu}{d\lambda}\) é o vetor tangente à curva.  

\textbf{Interpretação:}  
O vetor não “gira” em relação à curvatura ao longo da trajetória.

\subsection*{Relação com Geodésicas}

A geodésica é a curva cujo vetor tangente se transporta paralelamente:

\[
u^\nu \nabla_\nu u^\mu = 0 \quad \Leftrightarrow \quad \frac{d^2 x^\mu}{d\lambda^2} + \Gamma^{\mu}_{\ \nu\rho} u^\nu u^\rho = 0
\]

Ou seja, equações das geodésicas = condição de transporte paralelo do vetor velocidade.

\section*{Símbolos de Christoffel (Conexão de Levi-Civita)}

\begin{equation}
\Gamma^{\mu}_{\ \nu\rho} = \frac{1}{2} g^{\mu\sigma}
\left( \frac{\partial g_{\sigma\nu}}{\partial x^\rho} +
       \frac{\partial g_{\sigma\rho}}{\partial x^\nu} -
       \frac{\partial g_{\nu\rho}}{\partial x^\sigma} \right)
\end{equation}

\textbf{Legenda:}
\begin{itemize}
    \item $g_{\mu\nu}$ — métrica do espaço-tempo;
    \item $g^{\mu\sigma}$ — inversa da métrica;
    \item Derivadas parciais medem variações da métrica nos diferentes pontos.
\end{itemize}

\textbf{Destrinche:}  
\(\Gamma\) indica como a direção de uma linha reta muda devido à curvatura local.

\section*{Exemplos}

\subsection*{1. Espaço-Tempo Plano}

\[
ds^2 = -c^2 dt^2 + dx^2 + dy^2 + dz^2
\]

\[
\frac{d^2 x^\mu}{d\tau^2} = 0
\]

\textbf{Interpretação:} Movimento retilíneo e uniforme, sem força.

\subsection*{2. Espaço-Tempo Curvo (Schwarzschild)}

\[
ds^2 = -\left(1 - \frac{2GM}{c^2 r}\right)c^2 dt^2
+ \left(1 - \frac{2GM}{c^2 r}\right)^{-1} dr^2
+ r^2 d\Omega^2
\]

\textbf{Explicação:}  
- Tempo passa mais devagar perto da massa;  
- Espaço radial é “esticado”;  
- Geodésicas explicam órbitas e deflexão de luz.

\section*{Resumo}

| Conceito | Significado | Equação |
|-----------|-------------|---------|
| Conexão | Como comparar vetores em pontos diferentes | $\Gamma^{\mu}_{\ \nu\rho}$ |
| Derivada covariante | Derivada respeitando curvatura | $\nabla_\mu V^\nu = \partial_\mu V^\nu + \Gamma^{\nu}_{\ \mu\rho} V^\rho$ |
| Transporte paralelo | Vetor não gira ao longo da curva | $u^\nu \nabla_\nu V^\mu = 0$ |
| Geodésica | Curva cujo vetor tangente se transporta paralelamente | $u^\nu \nabla_\nu u^\mu = 0$ |

\end{document}
