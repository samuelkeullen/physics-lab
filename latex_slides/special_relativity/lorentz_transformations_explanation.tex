\documentclass[12pt,a4paper]{article}
\usepackage[T1]{fontenc}
\usepackage[utf8]{inputenc}
\usepackage[brazil]{babel}
\usepackage{amsmath,amssymb}
\usepackage{geometry}
\geometry{margin=2.5cm}

\title{Transformações de Lorentz}
\author{Samuel Keullen Sales}
\date{\today}

\begin{document}

\maketitle
%\tableofcontents
%\newpage

% Conteúdo principal começa aqui
\section{Transformações de Lorentz e Dilatação do Tempo}

\subsection{Notação e conceitos básicos}

\begin{itemize}
    \item $c$: velocidade da luz
    \item $\Delta t$: tempo medido pelo observador externo
    \item $\Delta \tau$: tempo próprio (no referencial do objeto)
    \item $\Delta x, \Delta y, \Delta z$: coordenadas espaciais
    \item $\Delta s$: intervalo espaço-temporal
\end{itemize}

As transformações de Lorentz substituem as transformações de Galileu porque, na relatividade, a velocidade da luz $c$ é a mesma para todos os observadores, independentemente do movimento relativo.

Como consequência, o intervalo espaço-temporal entre dois eventos,
\begin{equation}
    \Delta s^2 = c^2 \Delta t^2 - \Delta x^2 - \Delta y^2 - \Delta z^2,
\end{equation}
é \textbf{invariante}, ou seja, todos os observadores medem o mesmo valor, mesmo que seus tempos e posições individuais sejam diferentes.

No referencial próprio de uma partícula ou objeto (onde ele está \emph{parado} espacialmente), o intervalo reduz-se a
\begin{equation}
    \Delta s^2 = c^2 \Delta \tau^2.
\end{equation}

Cada partícula ou objeto “viaja” no espaço-tempo com uma \emph{velocidade total} igual a $c$. O que muda entre observadores é como essa velocidade se distribui entre deslocamento espacial e avanço no tempo, gerando os efeitos de dilatação do tempo e contração do espaço.

Essa interdependência entre espaço e tempo é a razão pela qual falamos em \emph{espaço-tempo}, em vez de tratá-los separadamente como na física clássica.

\subsubsection*{Resumo intuitivo}
\begin{itemize}
    \item Luz constante $\rightarrow$ intervalo fixo $\rightarrow$ mistura de espaço e tempo $\rightarrow$ espaço-tempo.
    \item Movendo-se rápido: parte da “velocidade do espaço-tempo” vai para o espaço, sobra menos para o tempo $\rightarrow$ relógios em movimento parecem andar devagar.
\end{itemize}

\subsection{O Relógio de Luz: dilatação do tempo}

Considere dois espelhos, um em cima e outro embaixo, separados por uma distância $L$.

\paragraph{Tempo próprio do relógio (no referencial do próprio relógio)} 
\begin{equation}
    \Delta \tau = \frac{2L}{c}.
\end{equation}

Exemplo numérico: $L = 2$ m
\begin{equation}
    \Delta \tau = \frac{2 \cdot 2}{3\times10^8} \approx 1.33 \times 10^{-8}\text{ s}.
\end{equation}

\paragraph{Tempo medido por um observador externo}  
O relógio se desloca horizontalmente junto com o trem. A luz percorre uma trajetória diagonal, maior que o caminho vertical. Como a velocidade da luz $c$ é constante, o tempo medido pelo observador externo aumenta. Esta diferença está ligada à invariância do intervalo $\Delta s$:
\begin{equation}
    \Delta s^2 = c^2 \Delta t^2 - \Delta x^2 - \Delta y^2 - \Delta z^2 = c^2 \Delta \tau^2.
\end{equation}

\subsubsection*{Medindo a trajetória diagonal}

Distâncias envolvidas:
\begin{itemize}
    \item Vertical: $L$
    \item Horizontal: deslocamento do trem durante metade do ciclo (subida): $x = v \Delta t / 2$
    \item Hipotenusa (trajetória da luz): $d = \sqrt{L^2 + x^2} = \sqrt{L^2 + (v \Delta t / 2)^2}$
\end{itemize}

A luz percorre essa diagonal à velocidade $c$:
\begin{equation}
    c \frac{\Delta t}{2} = \sqrt{L^2 + \left(\frac{v \Delta t}{2}\right)^2}.
\end{equation}

Isolando $\Delta t$:
\begin{align}
    c^2 \frac{(\Delta t)^2}{4} &= L^2 + v^2 \frac{(\Delta t)^2}{4} \\
    (c^2 - v^2)\frac{(\Delta t)^2}{4} &= L^2 \\
    \Delta t &= \frac{2 L}{\sqrt{c^2 - v^2}} = \frac{\Delta \tau}{\sqrt{1 - v^2/c^2}}.
\end{align}

\subsubsection*{Exemplo numérico}

Velocidade do trem: $v = 0.8 c$  
Tempo próprio: $\Delta \tau \approx 1.33 \times 10^{-8}$ s  
Tempo medido pelo observador externo:
\begin{equation}
    \Delta t = \frac{1.33 \times 10^{-8}}{\sqrt{1 - 0.8^2}} \approx 2.22 \times 10^{-8}\text{ s}.
\end{equation}

Resultado: O tempo medido pelo observador externo é maior que o tempo próprio do relógio, confirmando a dilatação do tempo.

\subsubsection*{Fórmulas gerais}

\paragraph{Forma completa:}
\begin{align*}
\Delta \tau &= \frac{2 L}{c} \\
\Delta t &= \frac{\Delta \tau}{\sqrt{1 - v^2 / c^2}}
\end{align*}

\paragraph{Forma adimensional simplificada:}
\begin{align*}
\beta &= \frac{v}{c} \\
\Delta t &= \frac{\Delta \tau}{\sqrt{1 - \beta^2}}
\end{align*}

\subsubsection*{Aplicação com exemplo}

\begin{itemize}
    \item $L = 2$ m
    \item $v = 0.8 c$
\end{itemize}

\paragraph{Cálculos completos:}
\begin{align*}
2L &= 4 \\
c^2 &= 9 \times 10^{16} \\
v^2 &= 5.76 \times 10^{16} \\
c^2 - v^2 &= 3.24 \times 10^{16} \\
\sqrt{c^2 - v^2} &= 1.8 \times 10^8 \\
\Delta \tau &= \frac{4}{3\times 10^8} \approx 1.33 \times 10^{-8} \\
\Delta t &= \frac{1.33 \times 10^{-8}}{0.6} \approx 2.22 \times 10^{-8}
\end{align*}

\paragraph{Forma adimensional:}
\begin{align*}
\beta &= 0.8 \\
\Delta t &= \frac{1.33 \times 10^{-8}}{\sqrt{1 - 0.8^2}} \approx 2.22 \times 10^{-8}
\end{align*}

Conclusão: O relógio em movimento anda mais devagar em relação ao observador externo.

\section{Simultaneidade Relativa}

\subsection{Conceito}

Na física clássica, o tempo é considerado absoluto: todos os observadores concordam sobre a simultaneidade de eventos.  

Na relatividade restrita, isso não é mais verdade: dois eventos que são simultâneos em um referencial $S$ podem \textbf{não ser simultâneos} em outro referencial $S'$ que se move com velocidade $v$ em relação ao primeiro.

Em outras palavras, a simultaneidade relativa explica que, mesmo que eventos sejam simultâneos em um referencial específico, eles podem ocorrer em momentos diferentes em outro referencial com velocidade relativa distinta.

\subsection{Consequências práticas}

\begin{itemize}
    \item Não existe um tempo universal, como descrito por Galileu. O tempo depende das coordenadas e do tempo próprio de cada referencial, considerando a velocidade da luz como limite.
    \item Eventos simultâneos em um referencial podem ocorrer em ordens diferentes em outro, mas apenas para eventos separados por distâncias espaciais maiores que a luz pode percorrer.
    \item Essa relatividade da simultaneidade explica paradoxos aparentes, como o \emph{paradoxo dos gêmeos}, sem violar a causalidade: eventos dentro do cone de luz mantêm a ordem causal.
\end{itemize}

\subsection{Exemplo prático: o trem e os relâmpagos}

Considere a clássica situação:

\begin{itemize}
    \item Um trem se move rapidamente sobre os trilhos.
    \item Dois relâmpagos atingem simultaneamente as extremidades do trem (frente e traseira).
    \item Um observador na plataforma vê os dois relâmpagos ao mesmo tempo.
    \item Um observador dentro do trem, que se move na direção da frente do trem, verá primeiro o relâmpago da frente e depois o da traseira.
\end{itemize}

\paragraph{Explicação:}  
A luz do relâmpago da frente percorre menos distância até o observador dentro do trem. Ou seja, para ele, os eventos não são simultâneos, mesmo que para a pessoa na plataforma parecessem.  

\emph{Conclusão:} A simultaneidade depende do referencial do observador.

\subsection{Ligação com as fórmulas de Lorentz}

Considere dois eventos em $S$:
\[
t_1 = t_2 \quad \text{e} \quad x_1 \neq x_2
\]

No referencial $S'$, em movimento relativo, a transformação de Lorentz altera os tempos dos eventos:
\[
t_1' \neq t_2'
\]

Portanto, segundo a simultaneidade relativa, eventos simultâneos em $S$ não são simultâneos em $S'$, especialmente se estão separados por distâncias grandes o suficiente para a luz não alcançá-los instantaneamente.

\emph{Resumo:} A noção de "ao mesmo tempo" depende do referencial do observador e está diretamente relacionada às transformações de Lorentz.

\section{Derivação Conceitual das Transformações de Lorentz}

\subsection{Premissas Básicas}

\subsubsection{1. Linearidade}

Começamos assumindo a forma mais geral possível, linear, para a transformação entre dois referenciais inerciais $S$ e $S'$:

\[
x' = A x + B t, \quad t' = D x + E t
\]

Onde os coeficientes possuem interpretações físicas:

\begin{itemize}
    \item $A$: fator de escala espacial, indica como as distâncias se transformam entre referenciais.
    \item $B$: mistura tempo-espaço, mostra quanto o tempo em $S$ afeta a posição em $S'$.
    \item $D$: mistura espaço-tempo, mostra quanto a posição em $S$ afeta o tempo em $S'$.
    \item $E$: fator de escala temporal, indica como o tempo se dilata ou contrai entre os referenciais.
\end{itemize}

Esses coeficientes ainda são desconhecidos, mas serão determinados pelas condições físicas a seguir.

\subsubsection{2. Coincidência das Origens}

Quando os dois referenciais coincidem no instante inicial:
\[
S: x = 0, \ t = 0 \quad \Rightarrow \quad S': x' = 0, \ t' = 0
\]

Essa condição garante que não há deslocamento espacial nem atraso temporal inicial entre as origens.

\subsubsection{3. Constância da Velocidade da Luz}

A luz deve se propagar com a mesma velocidade $c$ em qualquer referencial inercial:

\[
x = c t \quad \Rightarrow \quad x' = c t'
\]

Essa exigência força espaço e tempo a se misturarem nas transformações, pois $x$ e $t$ não podem mais ser tratados como independentes.

\subsection{Aplicação da Condição da Luz}

Considerando a luz se movendo nos sentidos positivo e negativo:

\[
\text{Sentido positivo: } x = c t, \quad x' = c t' 
\]
\[
\text{Sentido negativo: } x = -c t, \quad x' = -c t'
\]

Substituindo nas equações gerais $x' = A x + B t$ e $t' = D x + E t$, obtemos:

\[
A = E, \quad B = c^2 D
\]

\begin{itemize}
    \item $A = E$: espaço e tempo se escalam pelo mesmo fator, refletindo a simetria entre medidas espaciais e temporais.
    \item $B = c^2 D$: a mistura entre tempo e espaço depende de $c^2$, conectando as unidades de tempo e distância.
\end{itemize}

\subsection{Introduzindo a Velocidade Relativa $v$}

A origem de $S'$ se move com velocidade $v$ vista de $S$, ou seja, $x = v t \implies x' = 0$.  

Substituindo em $x' = A x + B t$:

\[
0 = A (v t) + B t \quad \Rightarrow \quad B = - A v
\]

Portanto, o coeficiente $B$ depende da velocidade relativa entre os referenciais, mostrando explicitamente como a transformação muda com $v$.

\section{Montando as Equações Reduzidas}

\subsection{Transformações em termos de $A$ e $v$}

Com as relações determinadas anteriormente:
\[
B = -A v, \quad E = A, \quad D = \frac{B}{c^2} = -\frac{A v}{c^2},
\]

as transformações de Lorentz para uma dimensão ficam:

\[
x' = A (x - v t), \quad t' = A \left(t - \frac{v}{c^2} x \right)
\]

\begin{itemize}
    \item $x' = A (x - v t)$: transforma posições, descontando o deslocamento relativo $v t$.
    \item $t' = A \left(t - \frac{v}{c^2} x \right)$: transforma tempos, mostrando que o tempo depende da posição $x$, introduzindo a simultaneidade relativa.
    \item O termo $\frac{v}{c^2} x$ é o fator de mistura espaço-tempo, que faz o tempo "inclinar" com o espaço.
\end{itemize}

\subsection{Determinando o fator $A = \gamma$}

Usamos a invariância do intervalo espaço-temporal:

\[
x'^2 - c^2 t'^2 = x^2 - c^2 t^2
\]

Substituindo $x'$ e $t'$ e simplificando:

\[
A^2 \left(1 - \frac{v^2}{c^2}\right) = 1 \quad \Rightarrow \quad A = \frac{1}{\sqrt{1 - v^2/c^2}} = \gamma
\]

O fator $\gamma$, conhecido como \textbf{fator de Lorentz}, controla efeitos relativísticos:

\begin{itemize}
    \item Quanto maior $v$, maior $\gamma$; quando $v \to c$, $\gamma \to \infty$.
    \item Para $v \ll c$, $\gamma \approx 1$, recuperando as transformações de Galileu.
\end{itemize}

\subsection{Fórmulas finais (1D)}

\[
x' = \gamma (x - v t), \quad t' = \gamma \left(t - \frac{v}{c^2} x \right)
\]

\[
x = \gamma (x' + v t'), \quad t = \gamma \left(t' + \frac{v}{c^2} x' \right)
\]

\begin{itemize}
    \item $x', t'$: coordenadas e tempo medidos no referencial em movimento $S'$.
    \item $x, t$: coordenadas e tempo no referencial estacionário $S$.
    \item A simetria entre as formas direta e inversa mostra que o movimento relativo é recíproco.
\end{itemize}

\subsection{Interpretação geométrica e física}

\begin{itemize}
    \item As equações representam uma rotação no espaço-tempo de Minkowski, onde o tempo e o espaço se "misturam".
    \item As linhas de simultaneidade de $S'$ não são horizontais em $S$, ilustrando a simultaneidade relativa.
    \item O intervalo 
    \[
    s^2 = c^2 t^2 - x^2
    \] 
    permanece invariante, garantindo a preservação da causalidade.
\end{itemize}

\subsection{Resumo}

Ao exigir linearidade, coincidência das origens e constância da velocidade da luz, o espaço e o tempo devem se misturar.  
Essa mistura leva naturalmente às transformações de Lorentz e explica:

\begin{itemize}
    \item Dilatação do tempo
    \item Contração do comprimento
    \item Simultaneidade relativa
\end{itemize}

Tudo controlado pelo fator $\gamma$.

\section{Conceito do Fator de Lorentz e Transformações de Lorentz}

\subsection{Determinando o fator $\gamma$}

Usando a invariância do intervalo espaço-temporal:
\[
x'^2 - c^2 t'^2 = x^2 - c^2 t^2
\]

Substituindo as expressões lineares e simplificando:
\[
A^2 \left(1 - \frac{v^2}{c^2}\right) = 1 \quad \Rightarrow \quad A = \frac{1}{\sqrt{1 - v^2/c^2}} = \gamma
\]

\begin{itemize}
    \item $\gamma$ é o \textbf{fator de Lorentz}, responsável pela dilatação do tempo e contração do comprimento.
    \item Quanto maior $v$, maior $\gamma$.
    \item Quando $v \to c$, $\gamma \to \infty$.
    \item Para $v \ll c$, $\gamma \approx 1$, recuperando as transformações de Galileu.
\end{itemize}

\subsection{Fórmulas finais (1D)}

\textbf{Diretas:}
\[
x' = \gamma (x - v t), \quad t' = \gamma \left(t - \frac{v}{c^2} x \right)
\]

\textbf{Inversas:}
\[
x = \gamma (x' + v t'), \quad t = \gamma \left(t' + \frac{v}{c^2} x' \right)
\]

\textbf{Legenda:}
\begin{itemize}
    \item $x'$: coordenada de um evento no referencial em movimento $S'$
    \item $t'$: tempo medido por relógio no referencial em movimento $S'$
    \item $x, t$: medidas no referencial estacionário $S$
    \item A simetria direta/inversa mostra que o movimento relativo é recíproco
\end{itemize}

\subsection{Exemplo prático – cálculo passo a passo}

Evento no referencial $S$:
\[
x = 6.0 \times 10^8 \text{ m}, \quad t = 3.0 \text{ s}
\]

Referencial $S'$ movendo com $v = 0.8\,c$.

\textbf{Calcular $\gamma$:}
\[
\gamma = \frac{1}{\sqrt{1 - v^2/c^2}} = \frac{1}{\sqrt{1 - 0.8^2}} = \frac{1}{\sqrt{0.36}} \approx 1.6667
\]

\textbf{Aplicar fórmulas:}

\[
x' = \gamma (x - v t), \quad v = 0.8 \, c = 2.4 \times 10^8 \text{ m/s}
\]
\[
x' = 1.6667 \, (6.0 \times 10^8 - 2.4 \times 10^8 \cdot 3) = 1.6667 \, (-1.2 \times 10^8) \approx -2.0 \times 10^8 \text{ m}
\]

\[
t' = \gamma \left(t - \frac{v}{c^2} x \right) = 1.6667 \left(3 - \frac{2.4 \times 10^8 \cdot 6.0 \times 10^8}{(3.0 \times 10^8)^2} \right)
\]
\[
t' = 1.6667 (3 - 1.6) = 1.6667 \cdot 1.4 \approx 2.33 \text{ s}
\]

\textbf{Resultado final:}

\[
\begin{array}{c|c|c}
\text{Referencial} & x \,(\text{m}) & t \,(\text{s}) \\
\hline
S & 6.0 \times 10^8 & 3.0 \\
S' & -2.0 \times 10^8 & 2.33
\end{array}
\]

→ O evento ocorre antes e atrás no sistema em movimento — mistura espaço-tempo confirmada.

\subsection{Resumo e interpretação}

\begin{itemize}
    \item As transformações de Lorentz preservam a velocidade da luz e o intervalo espaço-temporal.
    \item Espaço e tempo se misturam: $x$ influencia $t'$ e $t$ influencia $x'$.
    \item O fator $\gamma$ determina dilatação do tempo e contração do comprimento.
    \item Para velocidades baixas ($v \ll c$), $\gamma \approx 1$, recuperando a física clássica.
\end{itemize}

\end{document}
