\documentclass[12pt]{article}
\usepackage[utf8]{inputenc}
\usepackage{amsmath, amssymb}
\usepackage{geometry}
\usepackage{lipsum}

\geometry{a4paper, margin=2.5cm}

\title{Tensor de Ricci e Escalar de Curvatura}
\author{Samuel Keullen Sales}
\date{\today}

\begin{document}

\maketitle

\section{História e Motivação}

O \textbf{Tensor de Ricci} foi introduzido por \textbf{Gregorio Ricci-Curbastro} e seu aluno \textbf{Tullio Levi-Civita} no final do século XIX, no contexto do desenvolvimento do cálculo tensorial, também conhecido como \textit{Ricci Calculus}.  
O objetivo inicial era criar ferramentas matemáticas para estudar a \textbf{geometria de variedades curvas} de forma covariante, ou seja, independente do sistema de coordenadas.

Mais tarde, Albert Einstein percebeu que o Tensor de Ricci era essencial para descrever \textbf{como a matéria curva o espaço-tempo} em sua teoria da \textbf{Relatividade Geral} (1915).

O \textbf{Escalar de Curvatura} surge naturalmente ao se contrair o Tensor de Ricci e fornece um valor escalar que representa a curvatura global em cada ponto de uma variedade. Ele é usado nas equações de Einstein para relacionar curvatura e matéria.

\section{Intuição Física}

\begin{itemize}
    \item O \textbf{Tensor de Ricci} mede a curvatura que afeta o volume de uma pequena bola de partículas em queda livre. Ele é derivado do Tensor de Riemann, resumindo as informações relevantes para volumes.
    \item O \textbf{Escalar de Curvatura} é a média da curvatura em todas as direções. Em termos físicos:
    \begin{itemize}
        \item $R > 0$: espaço curvado positivamente (como uma esfera)
        \item $R < 0$: espaço curvado negativamente (como um hiperboloide)
        \item $R = 0$: espaço plano
    \end{itemize}
\end{itemize}

\section{Definições Matemáticas}

\subsection{Tensor de Ricci}

O Tensor de Ricci $R_{\mu\nu}$ é obtido a partir do \textbf{Tensor de Riemann} $R^\rho_{\ \sigma\mu\nu}$ através da contração de um índice:

\[
R_{\mu\nu} = R^\rho_{\ \mu\rho\nu}
\]

\textbf{Legenda:}
\begin{itemize}
    \item $R_{\mu\nu}$: Tensor de Ricci (2 índices, simétrico)
    \item $R^\rho_{\ \sigma\mu\nu}$: Tensor de Riemann (4 índices)
    \item $\rho$: índice de contração (somado)
\end{itemize}

\subsection{Escalar de Curvatura}

O Escalar de Curvatura $R$ é obtido contraindo o Tensor de Ricci com a métrica $g^{\mu\nu}$:

\[
R = g^{\mu\nu} R_{\mu\nu}
\]

\textbf{Legenda:}
\begin{itemize}
    \item $R$: Escalar de Curvatura
    \item $g^{\mu\nu}$: Tensor métrico inverso
    \item $R_{\mu\nu}$: Tensor de Ricci
\end{itemize}

\section{Exemplos de Aplicação}

\subsection{Espaço plano (Minkowski)}

Métrica em 4D:
\[
g_{\mu\nu} = \text{diag}(-1, 1, 1, 1)
\]

Cálculo do Tensor de Ricci e Escalar de Curvatura:
\[
R_{\mu\nu} = 0, \quad R = 0
\]

\textbf{Interpretação:} Espaço-tempo plano, sem curvatura.

\subsection{Superfície esférica 2D}

Métrica em coordenadas esféricas $(\theta, \phi)$:
\[
ds^2 = r^2 d\theta^2 + r^2 \sin^2\theta\, d\phi^2
\]

Cálculo do Tensor de Ricci:
\[
R_{\theta\theta} = 1, \quad R_{\phi\phi} = \sin^2\theta
\]

Escalar de Curvatura:
\[
R = \frac{2}{r^2}
\]

\textbf{Interpretação:} Espaço curvo com curvatura positiva constante; geodésicas se convergem.

\section{Exercícios Resolvidos}

\subsection*{Exercício 1: Espaço plano 2D}

\textbf{Problema:} Calcule $R_{\mu\nu}$ e $R$ para o espaço plano 2D.

\textbf{Solução passo a passo:}
\begin{enumerate}
    \item Métrica do espaço plano 2D: $ds^2 = dx^2 + dy^2$
    \item Cálculo do Tensor de Riemann: todos os coeficientes $R^\rho_{\ \sigma\mu\nu} = 0$
    \item Tensor de Ricci: $R_{\mu\nu} = R^\rho_{\ \mu\rho\nu} = 0$
    \item Escalar de Curvatura: $R = g^{\mu\nu} R_{\mu\nu} = 0$
\end{enumerate}

\textbf{Resultado:} $R_{\mu\nu} = 0$, $R = 0$ (espaço plano)

\subsection*{Exercício 2: Superfície esférica de raio $r$}

\textbf{Problema:} Verifique que $R = 2/r^2$.

\textbf{Solução passo a passo:}
\begin{enumerate}
    \item Métrica: $ds^2 = r^2 d\theta^2 + r^2 \sin^2\theta\, d\phi^2$
    \item Tensor de Ricci:
    \[
    R_{\theta\theta} = 1, \quad R_{\phi\phi} = \sin^2\theta
    \]
    \item Escalar de Curvatura:
    \[
    R = g^{\theta\theta} R_{\theta\theta} + g^{\phi\phi} R_{\phi\phi}
    \]
    Onde $g^{\theta\theta} = 1/r^2$, $g^{\phi\phi} = 1/(r^2 \sin^2\theta)$
    \item Substituindo:
    \[
    R = \frac{1}{r^2} \cdot 1 + \frac{1}{r^2 \sin^2\theta} \cdot \sin^2\theta = \frac{2}{r^2}
    \]
\end{enumerate}

\textbf{Resultado:} Escalar de Curvatura positivo constante, indicando espaço curvo.

\end{document}
