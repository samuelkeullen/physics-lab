\documentclass[a4paper,12pt]{article}
\usepackage[utf8]{inputenc}
\usepackage[brazil]{babel}
\usepackage{amsmath,amssymb}
\usepackage{graphicx}
\usepackage{geometry}
\geometry{margin=2cm}

\title{Espaço-Tempo na Relatividade}
\author{Samuel Keullen Sales}
\date{\today}

\begin{document}
\maketitle

\section*{1. Origem da Ideia}

Antes de Einstein, acreditava-se que o \textbf{espaço} e o \textbf{tempo} eram entidades separadas e absolutas, conforme a física de Newton.  
O espaço era o palco fixo onde os corpos se moviam, e o tempo fluía igualmente para todos os observadores.

Com o avanço das equações de Maxwell, percebeu-se que a \textbf{velocidade da luz} $c$ é a mesma para todos os observadores.  
Essa constância levou Einstein, em 1905, a propor a \textbf{Relatividade Especial}, onde espaço e tempo se unem em uma entidade única: o \textbf{espaço-tempo}.

Poucos anos depois, Hermann Minkowski formalizou matematicamente essa ideia:
\begin{quote}
``A partir de agora, o espaço em si e o tempo em si estão condenados a desaparecer, e apenas uma união dos dois conservará uma realidade independente.''
\end{quote}

\section*{2. Espaço-Tempo: Conceito e Estrutura}

O espaço-tempo é uma estrutura de \textbf{quatro dimensões}, composta por três espaciais e uma temporal:
\[
x^{\mu} = (ct, x, y, z)
\]
onde $\mu = 0,1,2,3$.

Cada ponto do espaço-tempo representa um \textbf{evento}, algo que ocorre em determinado lugar e instante.  
A trajetória de uma partícula nesse espaço quadridimensional é chamada de \textbf{linha de mundo}.

\section*{3. Métrica de Minkowski}

Para medir “distâncias” entre eventos no espaço-tempo, utilizamos o \textbf{tensor métrico} $g_{\mu\nu}$, que no espaço-tempo plano (sem gravidade) é:
\[
g_{\mu\nu} =
\begin{pmatrix}
-1 & 0 & 0 & 0 \\
0 & +1 & 0 & 0 \\
0 & 0 & +1 & 0 \\
0 & 0 & 0 & +1
\end{pmatrix}
\]

O \textbf{intervalo espaço-temporal} é definido por:
\[
ds^2 = g_{\mu\nu}\, dx^{\mu} dx^{\nu} = -c^2 dt^2 + dx^2 + dy^2 + dz^2
\]

\textbf{Legenda:}
\begin{itemize}
    \item $ds^2$ — intervalo invariante entre dois eventos;
    \item $g_{\mu\nu}$ — métrica de Minkowski;
    \item $dx^{\mu}$ — pequenas variações nas coordenadas espaço-temporais.
\end{itemize}

Esse intervalo $ds^2$ é o mesmo para todos os observadores inerciais — ou seja, é \textbf{invariante de Lorentz}.

\section*{4. Intuição Física}

Na física clássica, todos concordam sobre o tempo e o espaço.  
Na relatividade, esses conceitos dependem do observador, mas o \textbf{intervalo espaço-temporal} é o mesmo para todos.

Isso significa que o espaço-tempo é a verdadeira “realidade geométrica”, e o espaço e o tempo são apenas projeções dessa estrutura quadridimensional.

\section*{5. Espaço-Tempo Curvo}

Einstein percebeu que a \textbf{gravidade} é uma manifestação da curvatura do espaço-tempo.  
Corpos massivos \textbf{distorcem} o espaço-tempo, e outros corpos se movem seguindo o caminho mais natural — as \textbf{geodésicas}.

A métrica em presença de gravidade depende da posição:
\[
g_{\mu\nu} = g_{\mu\nu}(x)
\]
e o intervalo geral é:
\[
ds^2 = g_{\mu\nu}(x) \, dx^{\mu} dx^{\nu}
\]

\section*{6. Exemplos}

\subsection*{Exemplo 1 — Espaço-tempo plano (sem gravidade)}
Um fóton se move de $(0,0)$ até $(ct,x)$:
\[
ds^2 = -c^2t^2 + x^2 = 0
\]
\textit{Conclusão:} a luz sempre percorre linhas de mundo nulas ($ds^2 = 0$).

\subsection*{Exemplo 2 — Espaço-tempo curvo (com gravidade)}
Próximo de uma massa $M$, a métrica é dada por (aproximação de Schwarzschild):
\[
ds^2 = -\left(1 - \frac{2GM}{rc^2}\right)c^2dt^2 
+ \left(1 - \frac{2GM}{rc^2}\right)^{-1}dr^2 + r^2 d\Omega^2
\]
onde $d\Omega^2 = d\theta^2 + \sin^2\theta\, d\phi^2$.

\textit{Conclusão:} o tempo flui mais lentamente em regiões de maior gravidade.

\section*{7. Tabela Resumo}

\begin{center}
\begin{tabular}{|c|c|c|}
\hline
\textbf{Etapa} & \textbf{Conceito} & \textbf{Transição} \\
\hline
Newton & Espaço e tempo absolutos & Independentes \\
\hline
Einstein (1905) & Espaço-tempo plano & Relatividade Especial \\
\hline
Einstein (1915) & Espaço-tempo curvo & Relatividade Geral \\
\hline
\end{tabular}
\end{center}

\section*{8. Exercícios}

\textbf{Exercício 1 — Espaço-tempo plano:}\\
Um observador vê um objeto mover-se a $0{,}6c$ durante $4\,s$.  
Calcule o intervalo $ds^2$ entre o início e o fim do movimento.

\[
ds^2 = -c^2(4)^2 + (0{,}6c \cdot 4)^2
\]

\textit{Analise o resultado e discuta se o intervalo é tipo-tempo, tipo-luz ou tipo-espaço.}

\vspace{0.4cm}

\textbf{Exercício 2 — Espaço-tempo curvo:}\\
No campo gravitacional terrestre ($\frac{2GM}{rc^2} \approx 1{,}4\times10^{-9}$), compare o tempo medido por dois relógios:
\begin{itemize}
    \item Um na superfície da Terra ($r = R_T$);
    \item Outro em um satélite a $r = R_T + 2{,}0\times10^7 \, \text{m}$.
\end{itemize}

Use a componente temporal da métrica de Schwarzschild para estimar a diferença de tempo e discuta o efeito da dilatação gravitacional.

\section*{Conclusão}

O conceito de espaço-tempo unifica as noções de espaço e tempo em uma única estrutura geométrica.  
Esse modelo substitui a ideia de força gravitacional por curvatura geométrica, abrindo caminho para as próximas etapas:
\begin{itemize}
    \item Tensor métrico geral $g_{\mu\nu}(x)$
    \item Geodésicas (trajetórias naturais)
    \item Tensor de curvatura de Ricci
    \item Equações de Einstein
\end{itemize}

\end{document}
