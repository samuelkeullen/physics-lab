\documentclass[12pt]{article}
\usepackage[utf8]{inputenc}
\usepackage{amsmath, amssymb, amsfonts}
\usepackage{geometry}
\geometry{a4paper, margin=2.5cm}

\title{Lagrangiana, Euler-Lagrange e Noether: Um Guia Destrinchado}
\author{Samuel Keullen Sales}
\date{\today}

\begin{document}

\maketitle

\section*{1. Lagrange}

Em um momento do sistema, calculamos a Lagrangiana:

\[
L = T - V
\]

onde $T$ é a energia cinética e $V$ é a energia potencial naquele instante. Cada $L$ representa o \emph{momento do sistema} naquele instante.

\section*{2. Ação}

Para avaliar um caminho completo do sistema, integramos $L$ ao longo do tempo:

\[
S = \int_{t_1}^{t_2} L(q, \dot{q}, t) \, dt
\]

Discretizando:

\[
S \approx (L_1 \cdot \Delta t) + (L_2 \cdot \Delta t) + \dots + (L_n \cdot \Delta t)
\]

Cada $L_i$ é a Lagrangiana naquele instante, e $\Delta t$ é o intervalo de tempo.

\section*{3. Perturbação da trajetória}

Criamos uma pequena perturbação na trajetória:

\[
q(t) \to q(t) + \delta q(t)
\]

Isso altera $T$ e $V$, gerando um Lagrangiano perturbado $L(q+\delta q, \dot{q}+\dot{\delta q})$ e uma ação perturbada $S[q+\delta q]$.

Comparando com a ação original:

\[
\delta S = S[q + \delta q] - S[q]
\]

Intuição prática: imagine desenhar um caminho e empurrar levemente com o dedo. Se a ação não muda, você está no caminho natural do sistema.

\section*{4. Condição do caminho físico (Euler-Lagrange)}

O caminho físico faz a ação ser estacionária:

\[
\delta S = 0
\]

Isso leva à equação de Euler-Lagrange:

\[
\frac{d}{dt} \frac{\partial L}{\partial \dot{q}} - \frac{\partial L}{\partial q} = 0
\]

Resolvendo, obtemos $q(t)$, a trajetória real do sistema.

\section*{5. Simetria e Noether}

Se o sistema possui simetria, ou seja, se a Lagrangiana é invariante sob uma transformação:

\[
L(q, \dot{q}, t) \text{ invariante}
\]

então, pelo Teorema de Noether, existe uma quantidade conservada.

A forma geral da quantidade conservada é:

\[
Q = \sum_i \frac{\partial L}{\partial \dot{q}_i} \delta q_i
\]

onde $\delta q_i$ representa a mudança infinitesimal da coordenada $q_i$ associada à simetria.

\subsection*{Como testar simetria na prática}

Para descobrir se energia, momento linear ou angular é conservado:

\begin{enumerate}
    \item Escolha a transformação que deseja testar (tempo, espaço, rotação).  
    \item Crie a Lagrangiana perturbada aplicando a transformação infinitesimal:  
    \[
    q_i \to q_i + \delta q_i
    \]  
    \item Calcule a variação da ação:  
    \[
    \delta S = \int_{t_1}^{t_2} \big( L(q_i + \delta q_i, \dot{q}_i + \dot{\delta q}_i) - L(q_i, \dot{q}_i) \big) dt
    \]  
    \item Se $\delta S = 0$, a Lagrangiana é invariante e existe uma quantidade conservada $Q$.  
\end{enumerate}

\subsection*{Exemplo prático de simetria}

\begin{itemize}
    \item Se $L$ não depende do tempo, a simetria temporal garante que a energia total é conservada: $E = T + V$.  
    \item Se $L$ não depende de uma coordenada espacial $x$, a simetria de translação garante que o momento linear é conservado.  
    \item Se $L$ é invariante sob rotações, o momento angular é conservado.
\end{itemize}

Intuição: se o sistema "não sente" a transformação, existe algo que permanece constante ao longo do tempo. Noether identifica exatamente essa quantidade.

\section*{6. Energia que sobra}

A energia total do sistema, quando existe simetria temporal, é:

\[
E = T + V
\]

Comparando com a Lagrangiana:

\[
L = T - V
\]

a diferença instantânea é:

\[
E - L = (T + V) - (T - V) = 2V
\]

Intuição prática:

\begin{itemize}
    \item $E = T + V$ é a energia total que realmente se conserva ao longo do tempo.  
    \item $L$ define a ação e a trajetória correta do sistema.  
    \item A diferença $E - L = 2V$ mostra matematicamente como a energia potencial contribui para a relação entre energia total e Lagrangiana em cada instante.  
\end{itemize}

\section*{7. Resumo mental}

\begin{itemize}
    \item $L = T - V$ → momento do sistema
    \item $\int L dt$ → ação
    \item Perturbação $q(t) \to q(t)+\delta q$ → $L$ perturbado → $\delta S$
    \item $\delta S = 0$ → Euler-Lagrange → caminho físico real
    \item Simetria → Noether → quantidade conservada:
    \[
    Q = \sum_i \frac{\partial L}{\partial \dot{q}_i} \delta q_i
    \]
    \item Energia conservada: $E = T + V$, diferença $E - L = 2V$
\end{itemize}

\section*{8. Observação sobre perturbação e simetria}

Se a Lagrangiana perturbada não muda a ação ($\delta S = 0$), a Lagrangiana é simétrica para aquela transformação. Isso garante, por Noether, que existe uma quantidade física constante. A simetria prática do sistema identifica exatamente qual grandeza se conserva.

\end{document}
