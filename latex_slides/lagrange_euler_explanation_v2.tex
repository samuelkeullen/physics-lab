\documentclass[a4paper,12pt]{article}
\usepackage[utf8]{inputenc}
\usepackage{amsmath, amssymb}
\usepackage{geometry}
\usepackage{physics}
\usepackage{siunitx}
\usepackage{hyperref}
\geometry{margin=2cm}

\title{Lagrange e Equações de Euler–Lagrange \\ \small{Resumo Teórico e Aplicações}}
\author{Samuel Keullen Sales}
\date{\today}

\begin{document}

\maketitle

\section*{Introdução}
O formalismo de Lagrange é uma reformulação elegante da mecânica clássica.
Em vez de tratar forças diretamente (como faz Newton com $F = ma$), 
Lagrange descreve o movimento de um sistema a partir de suas \textbf{energias} e \textbf{restrições}.

A grande vantagem é que ele permite resolver problemas complexos com coordenadas não-cartesianas,
sistemas com restrições e, futuramente, se generaliza facilmente para a Relatividade e a Mecânica Quântica.

\section*{1. Conceitos Fundamentais}

\subsection*{1.1 Lagrangiana}
A Lagrangiana é definida como:
\[
L = T - V
\]
onde:
\begin{itemize}
    \item $T$ é a energia cinética do sistema;
    \item $V$ é a energia potencial.
\end{itemize}

\noindent
\textbf{Forma expandida:}
\[
L(q_i, \dot{q_i}, t) = \frac{1}{2} m \dot{q_i}^2 - V(q_i)
\]
Essa expressão depende das coordenadas generalizadas $q_i$, suas derivadas temporais $\dot{q_i}$, e do tempo $t$.

---

\subsection*{1.2 Equação de Euler–Lagrange}
A condição para que o caminho físico seja aquele realmente seguido pelo sistema é obtida do \textbf{princípio da ação estacionária}:
\[
\delta S = 0, \quad \text{onde } S = \int_{t_1}^{t_2} L \, dt
\]
A partir dessa condição, obtém-se a equação de Euler–Lagrange:
\[
\frac{d}{dt}\left( \frac{\partial L}{\partial \dot{q_i}} \right) - \frac{\partial L}{\partial q_i} = 0
\]

\noindent
\textbf{Forma destrinchada:}
\[
\underbrace{\frac{\partial L}{\partial \dot{q_i}}}_{\text{momento conjugado}} 
\quad \xrightarrow{d/dt} \quad
\underbrace{\frac{d}{dt}\left( \frac{\partial L}{\partial \dot{q_i}} \right)}_{\text{variação temporal do momento}} 
\quad - \quad
\underbrace{\frac{\partial L}{\partial q_i}}_{\text{variação espacial da Lagrangiana}} = 0
\]
O resultado expressa o equilíbrio dinâmico do sistema.

---

\section*{2. Exemplos Fundamentais}

\subsection*{2.1 Partícula Livre}
Para uma partícula de massa $m$ movendo-se livremente:
\[
L = \frac{1}{2} m \dot{x}^2
\]
Aplicando Euler–Lagrange:
\[
\frac{d}{dt}\left( m \dot{x} \right) - 0 = 0 \Rightarrow m \ddot{x} = 0
\]
\textbf{Interpretação:} A aceleração é nula, portanto o movimento é retilíneo uniforme.
Esse é o caso mais simples — e é o ponto de partida para relatividade.

---

\subsection*{2.2 Massa em Mola (Oscilador Harmônico)}
Para uma mola ideal com constante $k$:
\[
L = \frac{1}{2} m \dot{x}^2 - \frac{1}{2} k x^2
\]

% --- Notas de memorização para energia cinética e potencial ---
\textbf{Notas para memorizar:}
\begin{itemize}
    \item \textbf{Energia Cinética ($T$)}:
        \begin{itemize}
            \item Massa em mola: $T = \frac{1}{2} m \dot{x}^2$ 
                → energia do movimento da massa ao longo do eixo da mola. 
                Pense: \textit{quanto mais rápido se move, maior T}.
        \end{itemize}
    \item \textbf{Energia Potencial ($V$)}:
        \begin{itemize}
            \item Massa em mola: $V = \frac{1}{2} k x^2$ 
                → energia armazenada na mola devido à deformação. 
                Pense: \textit{quanto mais a mola é comprimida/esticada, maior V}.
        \end{itemize}
\end{itemize}
\textbf{Resumo prático:} $T$ mede “energia de movimento” (liberdade), $V$ mede “energia de restrição” (força restauradora). O sistema oscila equilibrando essas duas energias.

Aplicando Euler–Lagrange:
\[
m \ddot{x} + kx = 0
\]
\textbf{Forma padrão:}
\[
\ddot{x} + \omega^2 x = 0, \quad \text{onde } \omega = \sqrt{\frac{k}{m}}
\]
O sistema oscila harmonicamente — solução:
\[
x(t) = A \cos(\omega t + \phi)
\]

---

\subsection*{2.3 Pêndulo Simples}
Para um pêndulo de massa $m$ e comprimento $l$:
\[
L = \frac{1}{2} m l^2 \dot{\theta}^2 - mgl(1 - \cos\theta)
\]

% --- Notas de memorização para energia cinética e potencial ---
\textbf{Notas para memorizar:}
\begin{itemize}
    \item \textbf{Energia Cinética ($T$)}:
        \begin{itemize}
            \item Pêndulo: $T = \frac{1}{2} m l^2 \dot{\theta}^2$ 
                → energia associada à velocidade angular da massa. 
                Pense: \textit{quanto mais rápido o pêndulo balança, maior T}.
        \end{itemize}
    \item \textbf{Energia Potencial ($V$)}:
        \begin{itemize}
            \item Pêndulo: $V = m g l (1 - \cos\theta)$ 
                → energia armazenada devido à altura relativa da massa. 
                Pense: \textit{quanto mais alto o pêndulo sobe, maior V}.
        \end{itemize}
\end{itemize}
\textbf{Resumo prático:} $T$ mede “energia de movimento” (liberdade), $V$ mede “energia de restrição” (força restauradora). O sistema oscila equilibrando essas duas energias.

Aplicando Euler–Lagrange:
\[
m l^2 \ddot{\theta} + mgl \sin\theta = 0
\]
ou
\[
\ddot{\theta} + \frac{g}{l}\sin\theta = 0
\]
Para pequenos ângulos ($\sin\theta \approx \theta$):
\[
\ddot{\theta} + \frac{g}{l}\theta = 0
\]
que é novamente o \textbf{oscilador harmônico simples}.

---

\section*{3. Interpretação Física e Conservações}

\subsection*{3.1 Quando o tempo não aparece explicitamente em $L$:}
\[
\frac{\partial L}{\partial t} = 0 \Rightarrow E = \dot{q_i}\frac{\partial L}{\partial \dot{q_i}} - L = \text{constante}
\]
\textbf{Interpretação:} Conservação da energia total.

\subsection*{3.2 Quando uma coordenada $q_i$ não aparece em $L$:}
\[
\frac{\partial L}{\partial q_i} = 0 \Rightarrow \frac{d}{dt}\left( \frac{\partial L}{\partial \dot{q_i}} \right) = 0
\]
\textbf{Interpretação:} Conservação do momento conjugado (momento linear ou angular, dependendo do sistema).

---

\section*{4. Perguntas Conceituais e Respostas}

\begin{enumerate}
    \item \textbf{O que representa fisicamente a Lagrangiana $L = T - V$?}\\
    Representa a diferença entre a energia de movimento e a energia de restrição do sistema. Minimizar a ação associada a $L$ gera as equações de movimento.

    \item \textbf{Por que usamos coordenadas generalizadas $q_i$?}\\
    Porque sistemas com restrições têm menos graus de liberdade do que o espaço 3D total; $q_i$ descreve apenas o essencial.

    \item \textbf{O que ocorre se $L$ não depende de $t$ explicitamente?}\\
    A energia total é conservada.

    \item \textbf{O que ocorre se $L$ não depende de uma coordenada $q_i$?}\\
    O momento conjugado correspondente é conservado.

    \item \textbf{Como Lagrange se relaciona com Newton?}\\
    Ambos produzem as mesmas equações de movimento; Lagrange o faz via energia e coordenadas generalizadas, sem lidar com forças diretamente.
\end{enumerate}

---

\section*{5. Conclusão}
O formalismo de Lagrange e as equações de Euler–Lagrange formam o \textbf{núcleo conceitual} de toda a física moderna:
\begin{itemize}
    \item Na \textbf{Relatividade}, a Lagrangiana incorpora o espaço-tempo de Minkowski;
    \item Na \textbf{Mecânica Quântica}, ela se transforma em uma densidade de campo na formulação de Feynman;
    \item E com o \textbf{Teorema de Noether}, conecta simetrias a leis de conservação.
\end{itemize}

Estudar Lagrange é o primeiro passo sólido rumo à Física Teórica Moderna.

\end{document}
