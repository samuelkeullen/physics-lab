\documentclass[12pt]{article}
\usepackage{amsmath, amssymb, siunitx}

\title{Pendulo Simples: Período Exato e Aproximado}
\author{}
\date{}

\begin{document}

\maketitle

\section*{Enunciado}

Um pêndulo simples tem comprimento \(l = 0.5\,\text{m}\) e é solto com ângulo inicial \(\theta_0 = 10^\circ\), sem velocidade inicial (\(\dot{\theta}_0 = 0\)). Utilize \(g = 9.81\,\text{m/s}^2\).

\begin{enumerate}
    \item Determine a equação do movimento usando Lagrangiana / Euler–Lagrange.
    \item Faça a aproximação de pequeno ângulo e calcule o período \(T_\text{aprox}\).
    \item Calcule o período exato \(T_\text{exato}\) usando a série para \(K(k)\) e estime o erro percentual da aproximação linear.
    \item Avalie se a hipótese de pequeno ângulo é aceitável (\(<1\%\) de erro).
\end{enumerate}

\section*{a) Equação do movimento via Lagrangiana}

A energia cinética do pêndulo é:
\[
T = \frac{1}{2} m (l \dot{\theta})^2 = \frac{1}{2} m l^2 \dot{\theta}^2
\]

A energia potencial é:
\[
V = m g l (1 - \cos\theta)
\]

A Lagrangiana é:
\[
L = T - V = \frac{1}{2} m l^2 \dot{\theta}^2 - m g l (1 - \cos\theta)
\]

Aplicando a equação de Euler-Lagrange:
\[
\frac{d}{dt} \frac{\partial L}{\partial \dot{\theta}} - \frac{\partial L}{\partial \theta} = 0
\]

\[
\frac{d}{dt} (m l^2 \dot{\theta}) + m g l \sin\theta = 0
\]

\[
\boxed{\ddot{\theta} + \frac{g}{l} \sin\theta = 0}
\]

---

\section*{b) Aproximação de pequeno ângulo}

Para \(\theta \ll 1\), \(\sin\theta \approx \theta\):
\[
\ddot{\theta} + \frac{g}{l} \theta = 0
\]

Período aproximado:
\[
T_\text{aprox} = 2 \pi \sqrt{\frac{l}{g}}
\]

\textbf{Cálculo:}
\[
\frac{l}{g} = \frac{0.5}{9.81} \approx 0.05097
\]
\[
\sqrt{\frac{l}{g}} = \sqrt{0.05097} \approx 0.22576
\]
\[
T_\text{aprox} = 2 \pi \cdot 0.22576 \approx 1.4185\,\text{s}
\]

---

\section*{c) Período exato via série de \(K(k)\)}

\subsection*{Passo 1: Conversão para radianos}

\[
\theta_0 = 10^\circ \cdot \frac{\pi}{180} \approx 0.17453\,\text{rad}
\]

Dividindo por 2:
\[
\frac{\theta_0}{2} = 0.08725
\]

\subsection*{Passo 2: Cálculo de \(k\)}

\[
k = \sin\left(\frac{\theta_0}{2}\right) \approx \sin(0.08725) \approx 0.0871
\]

\[
k^2 = 0.0075
\]

\subsection*{Passo 3: Série de \(K(k)\)}

\[
K(k) = \frac{\pi}{2} \left( 1 + \frac{1}{4} k^2 + \frac{9}{64} k^4 + \frac{25}{256} k^6 + \frac{1225}{16384} k^8 \right)
\]

\textbf{Cálculo detalhado:}
\[
\frac{1}{4} k^2 = 0.25 \cdot 0.0075 = 0.001875
\]
\[
\frac{9}{64} k^4 = 0.140625 \cdot (0.0075)^2 = 7.91 \times 10^{-6}
\]
\[
\frac{25}{256} k^6 = 0.09765625 \cdot (0.0075)^3 \approx 4.11 \times 10^{-8}
\]
\[
\frac{1225}{16384} k^8 \approx 0.0748 \cdot (0.0075)^4 \approx 2.36 \times 10^{-10}
\]

Soma:
\[
1 + 0.001875 + 7.91\times10^{-6} + 4.11 \times 10^{-8} + 2.36\times10^{-10} \approx 1.00187795
\]

\[
K(k) \approx \frac{\pi}{2} \cdot 1.00187795 \approx 1.5737
\]

---

\subsection*{Passo 4: Período exato}

\[
T_\text{exato} = 4 \sqrt{\frac{l}{g}} K(k)
\]

\[
4 \cdot 0.22576 \cdot 1.5737 \approx 1.421\,\text{s}
\]

---

\section*{d) Comparação e validação da hipótese de pequeno ângulo}

\textbf{Diferença em segundos:}
\[
\Delta T = T_\text{exato} - T_\text{aprox} = 1.421 - 1.4185 \approx 0.0025\,\text{s}
\]

\textbf{Erro percentual:}
\[
\text{Erro} = \frac{\Delta T}{T_\text{exato}} \cdot 100 \approx \frac{0.0025}{1.421} \cdot 100 \approx 0.16\%
\]

\textbf{Conclusão:}  
O erro é menor que 1\%, logo a aproximação de pequeno ângulo é aceitável para \(\theta_0 = 10^\circ\).

\end{document}
