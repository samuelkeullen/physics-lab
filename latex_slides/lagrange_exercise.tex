\documentclass[a4paper,12pt]{article}
\usepackage{amsmath}
\usepackage{amssymb}
\usepackage{siunitx}
\usepackage{geometry}
\geometry{margin=2.5cm}
\begin{document}

\section*{Exercício – Lagrangiano de um bloco preso a uma mola em plano inclinado}

\textbf{Enunciado:}  
Um bloco de massa $m = 2\,\mathrm{kg}$ está preso a uma mola de constante elástica $k = 50\,\mathrm{N/m}$ em um plano inclinado de $30^\circ$.  
Considerando $\sin(30^\circ) = 0.5$ e $g = 9.8\,\mathrm{m/s^2}$, deseja-se:

1. Analisar a hipótese de movimento $x(t) = 0.1\,t^2$ (m).  
2. Calcular as energias cinética e potenciais.  
3. Determinar a Lagrangiana $L = T - V$.  
4. Verificar se a equação de Euler–Lagrange é satisfeita.  
5. Recalcular aplicando a condição inicial do ponto de equilíbrio $x(0) = -0.196$ m e $\dot{x}(0) = 0$.

---

\section*{Dados iniciais}

\begin{align*}
m &= 2\,\mathrm{kg} \\
k &= 50\,\mathrm{N/m} \\
\theta &= 30^\circ \quad \Rightarrow \quad \sin\theta = 0.5 \\
g &= 9.8\,\mathrm{m/s^2}
\end{align*}

Hipótese de teste: $x(t) = 0.1\,t^2$ (m)  
Tempos para avaliar: $t = 0,\,1,\,2,\,3$

---

\section*{Velocidade}

\[
\dot{x}(t) = \frac{d}{dt}(0.1t^2) = 0.2t
\]

\begin{align*}
t_0 &: 0.2(0) = 0 \\
t_1 &: 0.2(1) = 0.2 \\
t_2 &: 0.2(2) = 0.4 \\
t_3 &: 0.2(3) = 0.6
\end{align*}

---

\section*{Aceleração}

\[
\ddot{x}(t) = \frac{d}{dt}(0.2t) = 0.2
\]

\begin{align*}
t_0 &: 0.2 \\ 
t_1 &: 0.2 \\ 
t_2 &: 0.2 \\ 
t_3 &: 0.2
\end{align*}

---

\section*{Energia Cinética}

\[
T = \frac{1}{2} m \dot{x}^2
\]

\begin{align*}
t_0 &: \tfrac{1}{2}(2)(0)^2 = 0 \\
t_1 &: \tfrac{1}{2}(2)(0.2)^2 = 0.04 \\
t_2 &: \tfrac{1}{2}(2)(0.4)^2 = 0.16 \\
t_3 &: \tfrac{1}{2}(2)(0.6)^2 = 0.36
\end{align*}

---

\section*{Energia Potencial Gravitacional}

\[
V_g = m g \sin\theta \; x(t)
\]

\begin{align*}
t_0 &: 2(9.8)(0.5)(0.1 \cdot 0^2) = 0 \\
t_1 &: 2(9.8)(0.5)(0.1 \cdot 1^2) = 0.98 \\
t_2 &: 2(9.8)(0.5)(0.1 \cdot 2^2) = 3.92 \\
t_3 &: 2(9.8)(0.5)(0.1 \cdot 3^2) = 8.82
\end{align*}

---

\section*{Energia Potencial da Mola}

\[
V_s = \frac{1}{2} k (x(t))^2
\]

\begin{align*}
t_0 &: \tfrac{1}{2}(50)(0.1 \cdot 0^2)^2 = 0 \\
t_1 &: \tfrac{1}{2}(50)(0.1 \cdot 1^2)^2 = 0.25 \\
t_2 &: \tfrac{1}{2}(50)(0.1 \cdot 2^2)^2 = 4.00 \\
t_3 &: \tfrac{1}{2}(50)(0.1 \cdot 3^2)^2 = 20.25
\end{align*}

---

\section*{Energia Potencial Total}

\[
V = V_g + V_s
\]

\begin{align*}
t_0 &: 0 + 0 = 0 \\
t_1 &: 0.98 + 0.25 = 1.23 \\
t_2 &: 3.92 + 4.00 = 7.92 \\
t_3 &: 8.82 + 20.25 = 29.07
\end{align*}

---

\section*{Lagrangiana}

\[
L = T - V
\]

\begin{align*}
t_0 &: 0 - 0 = 0 \\
t_1 &: 0.04 - 1.23 = -1.19 \\
t_2 &: 0.16 - 7.92 = -7.76 \\
t_3 &: 0.36 - 29.07 = -28.71
\end{align*}

---

\section*{Equação de Euler–Lagrange}

\[
L = \tfrac{1}{2}m\dot{x}^2 - \left(m g \sin\theta\, x + \tfrac{1}{2}k x^2\right)
\]

A equação de movimento resultante é:

\[
m \ddot{x} + kx + m g \sin\theta = 0
\]

Substituindo $x(t) = 0.1t^2$ e $\ddot{x} = 0.2$, observa-se que o termo não se anula, logo a hipótese não satisfaz a equação.

---

\section*{Condição de Equilíbrio Estático}

Devido à presença do termo gravitacional, o equilíbrio não ocorre em $x = 0$, mas sim em:
\[
x_{eq} = - \frac{m g \sin\theta}{k} = -\frac{9.8}{50} = -0.196\,\mathrm{m}
\]

Vamos aplicar a condição inicial $x(0) = -0.196$ e $\dot{x}(0) = 0$ e revalidar os termos.

---

\section*{Nova avaliação (condição inicial $x(0)=-0.196$)}

\begin{align*}
x(t) &= -0.196 \quad \text{(constante)} \\
\dot{x}(t) &= 0 \\
\ddot{x}(t) &= 0
\end{align*}

\textbf{Energia Cinética:}
\[
T = \tfrac{1}{2} m \dot{x}^2 = 0
\]

\textbf{Potencial Gravitacional:}
\[
V_g = m g \sin\theta\, x = 2(9.8)(0.5)(-0.196) = -1.9208
\]

\textbf{Potencial da Mola:}
\[
V_s = \tfrac{1}{2} k x^2 = \tfrac{1}{2}(50)(-0.196)^2 = 0.9604
\]

\textbf{Energia Potencial Total:}
\[
V = V_g + V_s = -1.9208 + 0.9604 = -0.9604
\]

\textbf{Lagrangiana:}
\[
L = T - V = 0 - (-0.9604) = 0.9604
\]

---

\section*{Verificação da Equação de Movimento}

\[
m\ddot{x} + kx + m g \sin\theta = 0
\]

Substituindo $x=-0.196$, $\ddot{x}=0$:
\[
2(0) + 50(-0.196) + 2(9.8)(0.5) = -9.8 + 9.8 = 0
\]

\textbf{Resultado:}  
A equação é exatamente satisfeita (resíduo = 0).  
Logo, $x(t) = -0.196$ é uma solução estática estável, com energia constante no tempo.

---

\section*{Conclusão}

A hipótese inicial $x(t)=0.1t^2$ não satisfaz a equação de movimento, pois o termo da força resultante não se anula.  
Ao aplicar a condição de equilíbrio $x(0)=-0.196$ e $\dot{x}(0)=0$, obtém-se uma solução estacionária ($x(t)$ constante) que satisfaz exatamente a equação de Euler–Lagrange e a equação de movimento.

\end{document}
